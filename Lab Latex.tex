\documentclass [a4paper, 10pt, twocolumn] {article}

\usepackage {geometry}
\usepackage{graphicx, caption}
\usepackage{float}

\geometry{top=8mm}
\geometry{inner=27mm}
\geometry{outer=26mm}

\begin {document}
\author {\large {Alena Cherkas, Aliaksandr Kupo} \\ {\textit {Francisk Skorina}} \\ {\textit{Gomel State University}} \\{Gomel, Belarus} \\{Email: \large{alenacherkas.iss@gmail.com, kupo@gsu.by}}}
\date {}
\title {\textbf {OSTIS Technology Integration with Third-party NLP Services}}
\maketitle {} 
\textbf{\textit{Abstract}—This article is devoted to the integration of
OSTIS Technology with third-party services based on neural
networks. As an example of integration, the task of keeping
minutes of meetings is considered, the solution of which is
based on transcription and summarization. Transcription
is carried out using the Whisper service, summarization —
retext.ai.} 

\textbf{\textit{Keywords}—OSTIS Technology integration; neural networks; transcription task; summarization task; Natural
Language Processing.}
\\\\
\centerline{\large{I. I\small{NTRODUCTION}}}
\\

After the appearance of ChatGPT, its analogues and
other services based on neural networks in the general
availability, a large number of opportunities appeared in
the professional activities of representatives of different
professions, scientific activities, and personal life. In
particular, the scope of application of OSTIS technology
has expanded [1].

Integration of OSTIS Technology with third-party
services based on neural networks allows you to create
universal components adapted to reuse. Thus, for example,
when using ChatGPT, it becomes possible not only to
get an answer to a given question, but also to save it by
formalization, that is, writing on SC-code [2], thereby
replenishing the global knowledge base. After that, the
system using this knowledge base will be able to process
the information received to solve other tasks [3].

Thus, when integrated with ChatGPT or its analogues,
the Nika dialog platform, operating on the basis of OSTIS,
can access the advantages of dialog systems representing
a neural network, while maintaining the advantages of
OSTIS.

ChatGPT can also generate insufficiently correct responses. If the user does not have enough knowledge in
the field of the question being asked, then it is almost
impossible to distinguish truth from lies in the neural
network response. This reduces the quality of work
with the service and its capabilities. The integration of
ChatGPT with OSTIS allows you to identify incorrect
information in the responses of the service.

The task of integrating ChatGPT with OSTIS, namely
the Nika dialog system, has already been solved. However,
OSTIS Technology can be integrated with services of
completely different orientation [1], working on the basis
of neural networks. Thus, it is possible to optimize their
work, expand functionality, and reduce the number of
shortcomings [4].
\\\\
\centerline{\large{II. O\small{VERVIEW OF EXISTING APPROACHES}}}
\\

The possibility of integration is demonstrated by the
example of to creating a meeting protocol. To do this, the
tasks of transcription and summarization will be solved.
To convert an audio recording into a text format, there
are many services based on neural networks. One of the
available services is Whisper from OpenAI.

Whisper is an Automatic Speech Recognition (ASR)
system trained on more than 650 thousand hours of
multilingual and multitasking controlled data collected
from the Internet. The developers of the service note that
using such a large and diverse data set leads to increased
resistance to accents, background noise and technical
language. In addition, Whisper allows you to perform
transcription in several languages, as well as translation
from these languages into English.

Whisper is built on the transformer architecture, stacking encoder blocks and decoder blocks with the attention
mechanism propagating information between both. It
will take the audio recording, split it into 30-second
chunks and process them one by one. For each 30-second
recording, it will encode the audio using the encoder
section and save the position of each word said, and
leverage this encoded information to find what was said
using the decoder.

The decoder will predict what user call tokens from all
this information, which are basically each words being
said. Then, it will repeat this process for the next word
using all the same information as well as the predicted
previous word, helping it guess the next one that would
make more sense.

The overall architecture is a classic encoder-decoder
that is similar to GPT-3 and other language models.
OpenAI open-sourced their code and everything instead
of an API, so everyone can use Whisper as a pre-trained
foundation architecture to build upon and create more
powerful models for his necessities.
\begin{figure}[H]
	\center{\includegraphics[width=1.0\linewidth]{draw1.png}}
	\caption*{Figure 1. Whisper architecture [5]}
\end{figure}

Transcription of audio into text takes place in several
stages. The audio signal that is recorded on the device may
contain noises, echoes, pauses and other acoustic features
that make speech recognition difficult. Therefore, before
transcription, the audio signal passes through various
filters and noise reduction algorithms that allow you to
get a cleaner sound.

After preprocessing the audio signal, speech recognition algorithms are superimposed on it. Usually these
algorithms are based on neural networks that are trained to
recognize speech in different conditions. These networks
are used to recognize phonemes, words and phrases in
audio.

After the neural network recognizes speech in audio,
it matches the received phrases with the text. Usually,
matching algorithms are used for this, which find the most
suitable variants of recognized phrases for each section
of audio.
\begin{figure}[H]
	\center{\includegraphics[width=1.0\linewidth]{draw2.png}}
	\caption*{Figure 2. Whisper multitasking format [6]}
\end{figure}


Neural networks cannot always correctly recognize
speech in audio, so Whisper uses various error-handling
algorithms that allow you to correct recognition errors
and improve the quality of transcription. The end result
is a textual representation of audio that can be used
for various tasks, such as content search and analysis,
automatic transcriptions, and subtitle creation.

The amount of pretraining speech recognition data
for a given language is very predictive of zero-shot
performance on that language in Fleurs.

For small models, performance on English speech
\begin{figure}[H]
	\center{\includegraphics[width=1.0\linewidth]{draw3.png}}
	\caption*{Figure 3. Correlation of pre-training supervision amount with downstream speech recognition}
		\center{\includegraphics[width=1.0\linewidth]{draw4.png}}
	\caption*{Figure 4. Multitask and multilingual transfer improves with scale [6]}
\end{figure}
recognition degrades when trained jointly in a multitask and multilingual setup. However, multilingual and
multitask models benefit more from scale and eventually
outperform models trained on English data only. At the
4-th picture 95 percents bootstrap estimate confidence
intervals are shown [6].

Another problem is creating an output from the existing
text version of the meeting. This will help to create a
meeting protocol.

Text Summarization is a natural language processing
(NLP) task that involves condensing a lengthy text
document into a shorter, more compact version while still retaining the most important information and meaning.
The goal is to produce a summary that accurately
represents the content of the original text in a concise
form.

During training, the neural network learns to identify
important phrases, sentences, and ideas in the full-length
text, and generate a shorter summary that accurately
captures the main points.

There are different approaches to text summarization,
including extractive methods that identify and extract important sentences or phrases from the text, and abstractive
methods that generate new text based on the content of
the original text.

One approach for text summarization is to use an
encoder-decoder architecture with an attention mechanism.
The encoder takes in the full-length text and encodes it
into a fixed-length vector representation. The decoder then
generates the summary based on this encoded representation. The attention mechanism allows the decoder to focus
on the most important parts of the encoded representation
while generating the summary.

A major limitation of this network is its incapacity to
extract significant contextual connections from extended
semantic sentences. When a lengthy text contains contextual relationships within its substrings, the basic seq2seq
model cannot recognize those contexts. As a result, the
model’s performance is somewhat compromised, leading
to reduced accuracy.
\begin{figure}[H]
	\center{\includegraphics[width=1.0\linewidth]{draw5.png}}
	\caption*{Figure 5. Encoder-decoder model [7]}
\end{figure}

The encoder network extracts or obtain features from
input data and products internal state vector or context
vectors, which summarizes the input sequence. In STM
networks, these vectors are referred to as hidden state
and cell state vectors. The decoder network interprets the
context vector produced by the encoder and generates the
output sequence. The final call of the encoder produces
the context vector that serves as the input to the first call
of the decoder network. The decoder then use this context
vector along with the initial states to start generating the
output sequence, and the outputs are considered for future
predictions.

Attention is a type of enhancement to the current
sequence-to-sequence network that addresses its limitations. It derives its name from its ability to highlight
important parts of a sequence. By considering only a
few relevant items in the input sequence, the output
sequence becomes conditional and is guided by weighted
constraints. These constraints are the contexts that receive
attention, which are subsequently used for training and
predicting the desired results.
\begin{figure}[H]
	\center{\includegraphics[width=1.0\linewidth]{draw6.png}}
	\caption*{Figure 6. Attention architecture [7]}
		\center{\includegraphics[width=1.0\linewidth]{draw7.png}}
	\caption*{Figure 7. Attention model [7]}
\end{figure}

The attention-based sequence-to-sequence model requires considerable computational power, but its performance is superior to that of the traditional sequence-tosequence model. Additionally, the model can demonstrate
how attention is focused on the input sequence when
predicting the output sequence. This is helpful in comprehending and identifying the specific input-output pairs
and the extent to which the model considers them.

The model aims to create a context vector that is
tailored for each output time step by selectively filtering
the input sequence [7].

Another approach is to use reinforcement learning,
where the neural network is trained to generate summaries
that maximize a reward function based on the quality of
the summary. This approach can be useful when there is
not enough labeled data available for supervised learning
[8].

The key to neural network text summarization is in
finding a balance between generating a summary that is
short and concise while still accurately conveying the
main ideas of the full-length text.
\end{document}