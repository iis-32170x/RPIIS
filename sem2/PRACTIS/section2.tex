\begin{SCn}
\scnheader{Нечаев А.В.ПримеТКПдРБПЗ-2023}
\scnheader{ применение технологии контекстного программирования для решения больших прикладных задач}
\scntext{аннотация}{В современной эпохе цифровых технологий и больших данных, разработка программного обеспечения становится всё более сложной и требовательной к ресурсам. Одним из подходов, позволяющих эффективно решать крупномасштабные задачи, является контекстное программирование. Эта статья рассматривает принципы и методологии контекстного программирования, их применение в различных сферах, от облачных вычислений до разработки встраиваемых систем. Особое внимание уделяется проблемам масштабируемости, безопасности и надежности систем, использующих контекстное программирование. Статья также обсуждает перспективы развития контекстного программирования, включая новые технологии и подходы, которые могут усилить его потенциал в решении сложных прикладных задач.}

\begin{scnrelfromlist}{ключевой знак}
    \scnitem{контекст}
    \scnitem{естественно-языковые сообщения}
\end{scnrelfromlist}

\scntext{цитата}{Применение технологии контекстного программирования для решения больших прикладных задач" затрагивает область разработки программного обеспечения, где контекстное программирование играет ключевую роль в создании масштабируемых и эффективных решений для сложных задач. Контекстное программирование — это подход, при котором поведение программы зависит от её окружения, включая состояние памяти, файловую систему, сетевые соединения и другие ресурсы. Этот подход позволяет программам адаптироваться к различным условиям выполнения, что особенно важно для больших и сложных систем, работающих в условиях переменных сред.}
\scntext{тип источника}{статья}
\begin{scnrelfromlist}{автор}
    \scnitem{Нечаев, А.В.}
\end{scnrelfromlist}
\newpage

\scnheader{Станкевич Л.А.КогниДС-2023}
\scnheader{когнитивные диалоговые системы}
\scntext{аннотация}{Лекция посвящена речевым диалоговым системам и их развитию. Рассмотрены существующие архитектуры диалоговых систем и обсуждены возможности их совершенствования на основе когнитивного подхода. Введение в
систему беседующего агента, управляющего диалогом через планирование
и реализующего накопление знаний в реальном времени, позволяет создавать когнитивные диалоговые системы, способные вести беседы подобно человеку, обучаться и адаптироваться к собеседнику.}

\begin{scnrelfromlist}{ключевой знак}
    \scnitem{естественный язык}
    \scnitem{тонтейнер}
\end{scnrelfromlist}


\scntext{цитата}{Когнитивные диалоговые системы имеют несколько обязательных компонент, решающих задачи: распознавания речи (Speech Recognition – SR),
понимание естественного языка (Natural Language Understanding – NLU),
управление диалогом (Dialog Management – DM), генерация естественного
языка (Natural Language Generator – NLG), синтез речи (Speech Synthesis –
SS). Основные задачи входят в набор обработки естественного языка (Natural
Language Processing – NLP), который включает задачи понимания текстовых
фраз естественного языка (NLU), а также задачу генерации фраз естественного языка (NLG).}
\scntext{тип источника}{статья}
\begin{scnrelfromlist}{автор}
    \scnitem{Станкевич, Л.А.}
\end{scnrelfromlist}

\newpage
\scnheader{Свягор Л..СеманАТЕЯ-2023}
\scnheader{семантический анализ текстов естественного языка: цели и средства}
\scntext{аннотация}{Семантический анализ текстов естественного языка — это процесс выявления и интерпретации значений и отношений между словами и фразами в тексте. Целью семантического анализа является понимание смысла текста, что позволяет машинам лучше понимать и интерпретировать человеческую речь. Семантический анализ широко применяется в области обработки естественного языка (NLP) для улучшения взаимодействия между человеком и машиной, включая поиск информации, автоматический перевод, суммаризацию текста и многое другое.}


\begin{scnrelfromlist}{ключевой знак}
    \scnitem{текст}
    \scnitem{анализ}
    \scnitem{семантическая сеть}
\end{scnrelfromlist}
\scntext{тип источника}{статья}
\begin{scnrelfromlist}{авторы}
    \scnitem{Свягор, Л.}
    \scnitem{Гладун, В.}
\end{scnrelfromlist}

\newpage
\scnheader{Григорьев А.С.МашинПЕЯпСЗ-2023}
\scnheader{машинное понимание естественного языка при составлении запросов к поисковой системе библиотеки}
\scntext{аннотация}{Семантический анализ текстов естественного языка — это процесс выявления и интерпретации значений и отношений между словами и фразами в тексте. Целью семантического анализа является понимание смысла текста, что позволяет машинам лучше понимать и интерпретировать человеческую речь. Семантический анализ широко применяется в области обработки естественного языка (NLP) для улучшения взаимодействия между человеком и машиной, включая поиск информации, автоматический перевод, суммаризацию текста и многое другое.}

\begin{scnrelfromlist}{ключевой знак}
    \scnitem{машинный анализ}
    \scnitem{понимание}
    \scnitem{смысл}
\end{scnrelfromlist}

\scntext{цитата}{В эпоху цифровых технологий и массового доступа к информации через Интернет, поисковые системы библиотек становятся неотъемлемой частью образовательного процесса и научных исследований. Однако, традиционные методы поиска информации ограничиваются простыми запросами и не учитывают сложность и многогранность естественного языка. Машинное понимание естественного языка (NLP) предлагает решение этих проблем, позволяя поисковым системам библиотек понимать и интерпретировать запросы пользователей, сформулированные естественным языком. Эта статья рассматривает принципы и методы NLP, применяемые для улучшения качества поиска информации в библиотечных системах, включая анализ настроений, распознавание именованных сущностей и семантический анализ. Особое внимание уделяется разработке алгоритмов и моделей, способных понимать намерения пользователя и предоставлять наиболее релевантные результаты поиска.}
\scntext{тип источника}{статья}
\begin{scnrelfromlist}{автор}
    \scnitem{Григорьев, А. С.}
\end{scnrelfromlist}
\end{SCn}