\begin{SCn}
\begin{small}
\scnheader{Кравченко Е.Г..МетодОКТП-2016ст}
\begin{scnrelfromlist}{ключевой знак}
    \scnitem{технологический процесс}
    \scnitem{функция желательности Харрингтона}
    \scnitem{свойства технологического процесса}
\end{scnrelfromlist}
\scntext{аннотация}{В статье рассматривается методика оценки качества технологических процессов по совокупности различных свойств технологического процесса (технических, экономических, эргономических и других), основанная на использовании безразмерного обобщенного показателя}
\scntext{цитата}{Технологический процесс, являющийся частью производственного процесса, в системе менеджмента качества (СМК) организаций относят к основным процессам жизненного цикла продукции}
\begin{scnindent}
\scnrelto{пояснение}{технологический процесс}
\end{scnindent}
\scntext{цитата}{Анализ отечественных и зарубежных литературных источников свидетельствует о том, что имеются различные подходы к оценке качества технологических процессов. Наиболее подходящим является подход, основанный на оценке процессов по шкале значимости Харрингтона, широко используется в системах менеджмента качества организаций для оценки их результативности.На основании предложенного подхода, качество технологических процессов необходимо оценивать по совокупности различных свойств.В основе такой методики лежит использование безразмерного обобщенного показателя, учитывающего всю совокупность необходимых потребителю свойств технологического процесса}
\begin{scnindent}
\scnrelto{пояснение}{функция желательности Харрингтона}
\end{scnindent}
\scnheader{Фурсенко С.Н..АвтомТП-2007кн }
\begin{scnrelfromlist}{ключевой знак}
    \scnitem{автоматизация}
    \scnitem{технологический процесс}
    \scnitem{сельское хозяйство}
    \scnitem{автоматический контроль}
    \scnitem{автоматическая защита}
    \scnitem{автоматическое управление}
\end{scnrelfromlist}
\scntext{аннотация}{Учебное пособие посвящено вопросам электроавтоматики первого уровня технологических процессов сельскохозяйственного производства.В издании показано значение и особенности автоматизации технологических процессов сельскохозяйственного производства, их влияние на синтез и разработку технологических требований к аппаратной части систем автоматического управления. Раскрыта технология проектирования систем автоматизации поточных линий. В том числе с использованием программируемых логических контроллеров в системах управления оборудованием. Описаны автоматические системы типовых технологических процессов сельскохозяйственного производства. Предназначено для студентов, широкого круга инженерно-технических работников.}
\scntext{цитата}{Автоматизация — применение технических средств, экономико-математических методов и систем управления, освобождающих человека частично или полностью от непосредственного участия в процессах получения, преобразования, передачи и использования энергии, материалов или информации. }
\begin{scnindent}
\scnrelto{пояснение}{автоматизация}
\end{scnindent}
\scntext{цитата}{Цель автоматизации — повышение производительности и эффективности труда, улучшение качества продукции, устранение человека от работы в условиях, опасных для здоровья. }
\begin{scnindent}
\scnrelto{пояснение}{автоматизация}
\end{scnindent}
\scntext{цитата}{В зависимости от функций, выполняемых специальными автоматическими устройствами, различают следующие основные виды автоматизации: автоматический контроль, автоматическую защиту и автоматическое 
управление.}
\begin{scnindent}
\scnrelto{пояснение}{автоматизация}
\end{scnindent}
\scntext{цитата}{Автоматический контроль включает автоматическую сигнализацию, 
измерение, сортировку и сбор информации.}
\begin{scnindent}
\scnrelto{пояснение}{автоматический контроль}
\end{scnindent}
\scntext{цитата}{Автоматическая защита представляет собой совокупность технических средств, которые при возникновении ненормальных и аварийных режимов прекращают контролируемый производственный процесс. Автоматическая защита тесно связана с автоматическим управлением и сигнализацией. 
Она воздействует на органы управления и оповещает обслуживающий персонал об осуществленной операции.}
\begin{scnindent}
\scnrelto{пояснение}{автоматическая защита}
\end{scnindent}
\scntext{цитата}{Автоматическое управление включает комплекс технических средств 
и методов по управлению, обеспечивающих пуск и остановку основных и 
вспомогательных устройств, безаварийную работу, соблюдение требуемых 
значений параметров в соответствии с оптимальным ходом технологического 
процесса.}
\begin{scnindent}
\scnrelto{пояснение}{автоматическое управление}
\end{scnindent}
\end{small}
\end{SCn}