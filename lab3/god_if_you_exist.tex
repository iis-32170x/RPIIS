\documentclass{article}
\usepackage[utf8]{inputenc}
\usepackage[top=2cm, bottom=1.8cm, left=2.5cm, right=2.5cm, footskip=15mm]{geometry}
\usepackage{multicol}
\usepackage{titlesec}
\usepackage{hyperref}

\title{Conceptual Design of Complex Integrated 
       
Systems}
\author{Victar Kochyn \\ \textit{Belarusian State University} \\ Minsk, Belarus\\ Email: \href{mailto:kochyn@bsu.by}{kochyn@bsu.by} 
   \and Alexander Kurbatsky \\ \textit{Belarusian State University} \\ Minsk, Belarus \\ Email: \href{mailto:kurbatski@bsu.by}{kurbatski@bsu.by} }
   \date{}
   
\begin{document} 
\maketitle

\begin{multicols}{2}
\setcounter{page}{33}
\textbf{\textit{Abstract-}The article describes the problems of digital
transformation of the Republic of Belarus. The ways of
solving the problem are proposed. A new approach to the
design and development of complex information systems
is proposed – the design of complex integrated systems.
The basic properties of complex integrated systems are
determined
}

\textbf{\textit{Keywords-} System design, complex systems, digital
transformation, system of systems, interoperability, integration}

\begin{center}
 \vspace{-3pt}
    {\large I. \textsc{Intoduction}}
\end{center}

Currently, it is difficult to overestimate the importance
of digitalization of key processes in any state. Digital
technologies make it possible tooptimize many management processes in the economy, healthcare, education,
and industry. The development of the modern economy is
largely based on the processes of digital transformation.
Until 2020, digitalization was an evolutionary process,
but the COVID-19 pandemic radically changed the role
and perceptionof digitalization in the state and society
and accelerated its pace. Digital technologies are now
essential for work, learning, entertainment, communication, shopping, and access to everything from health
services to culture. One of the important conditions for
successful implementation of the digital transformation
strategy is the development of new approaches to the
design and development of information systems. As such
an approach, it is proposed to use the design of complex
integrated systems [1], [2].
\begin{flushleft}
 {\textit{A.Digital transformation of the Republic of Belarus
}}   
\end{flushleft}


Despite the undoubted successes in the development of
the information and communication infrastructure of the
Republic of Belarus, the creation of individual elements
of e-government, it is premature to talk about significant
progress in the digitalization of the public sector of
Belarus for a number of reasons:
\vspace{-5pt}
\begin{itemize}
 \setlength\itemsep{-4pt}
    \item Many platforms and systems were originally developed to solve specific tasks and did not provide for
the possibility and necessity of integration, as well
as integration into the chain of industry, country
and supranational platforms. Often the developed
solutions were not integrated with each other.
\item Digitalization in our country has developed chaotically and sometimes uncontrollably from the point
of view of embedding in a single strategy of digital
transformation of the country.
\item Currently, there are practically no industry platforms
into which digital platforms of enterprises can be
integrated.
\item Many institutions and departments use proprietary
software to digitalize key processes. Often, when
using such software, enterprises have to adapt their
business processes to the functionality imposed by
the manufacturer, and not vice versa.
\item Comprehensive information security is not fully ensured. This problem includes both the development
of software and hardware solutions and security in
the information space.
\end{itemize}
The digital transformation of the economy and society
of Belarus should contribute to the achievement of the
following goals:
\begin{itemize}
\setlength\itemsep{-4pt}
\item Ensuring the digital sovereignty of the country
\item Creating conditions for the introduction of innovative solutions in the sphere of economy and society,
as well as for the integration processes of both
internal and external country digital platforms
\item Implementation of the import substitution strategy
in the field of digitalization of key processes of the
economy and society.
\item Creation of complex information security systems.
\item Creating conditions and guidelines for young people.
\end{itemize}
One of the significant factors of ensuring the sovereignty
of the state in cyberspace is the desire for independence
(sovereignty) ICT or more broadly digital sovereignty.
Securing digital sovereignty is becoming increasingly
difficult in a globalized world. At the same time, there
is currently no clear definition of the digital sovereignty
of the state. The author Ashmanov I.S. defines digital
sovereignty as the right of a state to determine its
information policy independently, manage infrastructure,
resources, ensure information security, etc. From the
point of view of staffing digital sovereignty, this process
involves high-quality personnel rotation (the arrival of
\begin{flushleft}
responsible specialists in the relevant ministries who
are thoroughly versed in the processes of digitalization
and IT industries), the creation of educational programs
at universities that train multidisciplinary specialists –
at the junction of IT technologies and public administration, public policy, innovative economy, the creation
of new jobs in the country, providing the state with
useful innovations in the field of artificial intelligence,
e-government, the Internet of things, electronic services,
new weapons systems, etc. From a technological point of
view, digital sovereignty is determined by the presence
of a sovereign complex of integrated and complementary
digital services and platforms in all key spheres of the life
of the state and society, including its own hardware base,
technological solutions in the field of content delivery,
as well as national digital platforms (social networks,
cloud storage, messengers, information storage services,
etc.).Thus, the digital sovereignty of a country is closely
linked to the ability to independently form an information
policy, manage information flows, ensure information
security, and ensure the storage and processing of digital
data regardless of external influence. Achieving these
goals requires increased expertise in the digital sphere.
The rules of behavior in the virtual space are being
actively discussed now. Probably, partly because of the
insufficient expert level, the world community as a
whole, and ours in particular, have not made much
progress on this issue. Currently, no State has been able
to fully achieve digital sovereignty. For example, China,
which has one of the most technologically advanced and
developed economies in the world, is heavily dependent
on a number of Western technologies (microchips, processors, etc.). The USA is a world leader in creating
ICT solutions. At the same time, a number of high-tech
industries have been transferred to other states.
\end{flushleft}
\vspace{+4pt}
\begin{flushleft}
    \textit{B. Conditions for the introduction of innovative solutions
in the sphere of economy and society}
\end{flushleft}
\vspace{-1pt}

Currently, many countries, including the Republic of
Belarus, are striving to create conditions for the introduction of digital innovative solutions in the sphere
of economy and society. As a rule, these issues are
regulated by various fundamental documents such as the
Digital Development Strategy, various state digitalization
programs, etc. In the Republic of Belarus, the issues of
innovative development are reflected in the Resolution
of the Council of Ministers of the Republic of Belarus
No. 66 dated February 2, 2021 on the approval of the
State Program "Digital Development of Belarus for 2021-
2025". This state program was adopted in order to ensure
the introduction of information and communication and
advanced production technologies in the branches of the
national economy and the sphere of life of society. The
program provides for the implementation of measures
for the introduction of digital innovative solutions in the 
sphere of economy and society. But for the successful
implementation of innovative solutions in the sphere of
economy and society, the following conditions must be
met:
\begin{itemize}
\setlength\itemsep{-2pt}
    \item Development and implementation of new approaches, methodologies in the field of design,
development, standardization and implementation of
industry and digital platforms.
\item Training of elite specialists in the field of development and implementation of innovative solutions in
the sphere of economy and society.
\end{itemize}
\vspace{-7pt}
The analysis of successful examples in the field of
digital transformation of the state shows that one of the
important conditions for the introduction of innovative
solutions in the sphere of economy and society was the
development of unified country approaches to the design,
development and implementation of innovative solutions.
When developing such solutions, it is advisable to use
the experience of leading countries, the existing level of
digitalization of the country, as well as the conditions and
features of the development of the economy and society
of the Republic of Belarus.
\vspace{+10pt}
\begin{flushleft}
    \textit{C. Approaches to system design}
\end{flushleft}
\vspace{+10pt}


The rapid development of global networks in the
late 90s - early 00s, primarily the Internet, created the
prerequisites for a sharp increase in the needs for various
information systems, in fact, the process of their creation
began to be massive. This was due to the massive introduction of computer technology in various spheres of
government and society, the development of data transmission networks. At the initial stage, digitization of existing documents and automation of individual processes
took place. One of the first directions of automation of
business processes was the development of information
systems for managing individual processes of enterprises,
such as automation of accounting, personnel accounting,
material values, etc. As information technologies were
introduced into production and business processes, the
complexity of information systems and services grew.
For this reason, approaches to the design and development of information systems have changed. The theoretical model of culture is a kind of coordinate
system, a system of key concepts that real. Classical
approaches no longer allowed the effective development
and implementation of complex systems. New design
approaches were required that could take into account
the complexity of systems, the possibility of scaling,
integration with other systems. A separate scientific
and methodological discipline, system engineering, is
devoted to the issues of designing complex systems. As
the complexity of information systems and services grew,
not only approaches to system design evolved, but also
processes in the digital sphere. In general, it is possible to
identify the main processes of digitalization, which were
formed as digital technologies penetrated into various
spheres of the economy and society:
\vspace{10pt}

\begin{itemize}
\setlength\itemsep{-4.8pt}
    \item Automation. Currently, there are many definitions
of this process: from general conceptual definitions
to descriptions of specific processes of an enterprise
or organization. For example, in [3], the automation
process is defined as "a direction of scientific and
technological progress that uses self-regulating technical means and mathematical methods in order to
free a person from participating in the processes of
obtaining, converting, transmitting and using energy,
materials, products or information, or significantly
reducing the degree of this participation or the
complexity of the operations performed." According
to GOST [4], automation is the introduction of automatic means for the implementation of processes;
a system of measures aimed at increasing human
productivity by replacing part of this labor with the
work of machines. It is based on the use of modern
computer technology and scientific methods. In [5],
automation is described as the first stage on the
way to digital transformation, when human labor is
replaced by machine labor. Summarizing the considered approaches to the definition of the automation
process, it can be concluded that automation is
a business or production process that is digitized,
while there is no optimization or change in the
business or production processes themselves.
\item Computerization. The widespread introduction of
computer technology is closely connected with the
process of computerization. According to GOST [4],
computerization is the process of automating any
processes in any field of human activity through
the use of computers. In [6], it is defined that
computerization is the widespread introduction of
computers into various spheres of human activity
(for example, for the management of technology,
transport, energy, etc. production processes). In the
encyclopedia [7], computerization is described as
a process of expanded introduction of electronic
computing technology into all spheres of human
activity. Based on the results of the analysis of the
presented definitions, it can be concluded that computerization is the process of mass introduction of
personal computers for the purpose of full–scale use
of automation in production or business processes.
\item Informatization. Many authors associate the next
stage of development and implementation of digital technologies in the state and society with the
process of informatization. At the same time, there
are various definitions of this process in the literature. Thus, in [8] informatization is described
as an organizational, socio-economic, scientific and
technical process that provides conditions for the
formation and use of information resources and
the implementation of information relations. The
Law of the Republic of Belarus "On Information,
Informatization and Information Protection" [9] provides the following definition: informatization is
an organizational, socio–economic, scientific and
technical process that provides conditions for the
formation and use of information resources and the
implementation of information relations. According
to [10], informatization is an organizational, socioeconomic, scientific and technical process of creating favorable conditions for meeting information
needs, realizing the rights and freedoms of subjects of the information sphere, which is based on
the mass application of information systems and
technologies in all types of activities of individuals
and legal entities. The author [11] gives the following definition: informatization is an unprecedented
increase in the speed and quantity of production
and dissemination of information, as well as the
increased role of information processes, systems and
networks using ICT in society. Based on the results
of the analysis and generalization, it is possible to
define informatization as a scientific and technical
process for the creation and implementation of
information systems and services in various fields
of activity, characterized by the massive penetration
of information technologies into all spheres of theeconomy and society.
\end{itemize}
\vspace{-24pt}
\begin{itemize}
\item Digital transformation. Currently, many institutions,
departments, companies, and industries have developed a digital transformation strategy. However,
there are many definitions of this process in the
literature. As a rule, definitions of digital transformation are based on the size of the object of digital
transformation (institution, industry, country). Thus,
in [8], the authors define digital transformation as a
manifestation of qualitative, revolutionary changes,
consisting not only in individual digital transformations, but in a fundamental change in the structure of
the economy, in the transfer of value-added centers
to the sphere of building digital resources and endto-end digital processes. The following definition
is given in [12]: digital transformation is the process of introduction of digital technologies by an
organization, accompanied by optimization of the
control system of the main technological processes.
Digital transformation is designed to accelerate sales
and business growth or increase the efficiency of
organizations that are not purely commercial (for
example, universities and other educational institutions). In [13], the authors conclude that digital transformation is simultaneously aimed at improving existing business processes and creating
competitive advantages by changing and creating
new business processes within the enterprise. Based
on the results of the analysis, we will determine
that digital transformation is a process of integrated
\end{itemize}
\end{multicols}
\end{document}