\documentclass{article}
\usepackage{textcomp}
\usepackage[utf8]{inputenc}
\usepackage[left=17mm, top=17mm, right=17mm, bottom=5mm, nohead, nofoot]{geometry}
\usepackage{enumitem}
\usepackage{multicol}
\usepackage{titlesec}
\usepackage{graphicx}
\usepackage[russian]{babel}
\setlist[itemize]{itemsep=0pt, parsep=0pt, partopsep=0pt, topsep=0pt}
\setcounter{page}{285}
\titleformat{\section}{\large\centering\sc}{\thesection. }{0cm}{}[]
\begin{document}

\begin{center}
\LARGE\textbf{About Creation of the Intelligent Transportation
Control System in Railway Transport
}
\end{center}\   


\begin{center}
\begin{tabular}{c}
Aleksandr Erofeev \\
\textit{Belarusian State University of Transport} \\
Gomel, Belarus \\
Email: erofeev\string_aa@bsut.by \\
\end{tabular}
\end{center} 
\

\begin{multicols}{2}

\textbf{
  \textit{Abstract}—The relevance of the development of an intelligent system for managing the transportation process
is determined. The structure of the system construction
theory is given. The experience of developing automated
systems on the Belarusian railway is described and the
effectiveness of their implementation is evaluated. It has
been established that the main condition for the interaction
of automated systems with each other is the use of a single
ontology of the transportation process. It is indicated that
the OSTIS technology is an effective tool for describing the
process-object ontology of the transportation process. The
advantages and limitations of using OSTIS technology in
ITCS are established. 
}\vspace{0.5\baselineskip}

\textbf{\textit{Keywords}—Intelligent Transportation Control System,
Ontology, process-object approach, transportation process,
OSTIS technology 
}\vspace{0.5\baselineskip}

Automation of individual tasks of managing the transportation process (TP) was one of the first areas of
informatization of the railway transport activity. However, in modern conditions, the efficiency of previously
developed automated systems (AS) has decreased due
to significant fluctuations in the power and structure of
traffic flows and changes in the technologies of the transportation process.Further development of existing AS
has significant limitations: the exact mathematical model
of the object may be too complex or unconstructible;
changes in the external object environment lead to the
action on the object of a number of perturbations, which
are an additional source of uncertainty about the state
of the object; performance requirements can be loosely
formalized and inconsistent. It is proposed to overcome these shortcomings by moving from informationreference and settlement systems to intellectual ones.\vspace{0.5\baselineskip}


 
BelSUT has developed a theory for building an Intelligent Transportation Control System (ITCS), the use of
which in the development, implementation and operation
will increase the adaptability of transportation process
technologies to a changing operational environment,
solve new operational problems, ensure coordination and
continuity of control decisions, improve system manageability, which Together, it will ensure the efficient
functioning of the railway in the face of changes in
the volume and structure of traffic flows and optimize
the operating costs for the organization of transportation
activities [1],[2].


 
 A graphical interpretation of the methodology for
creating an ITCS is shown in “Fig. 1”.

 The creation of the ITCS is aimed at:
\begin{itemize}[noitemsep]
    \item implementation of a coordinated integrated transportation process management system (TPMS) using by all participants in this activity a single
digital model of the transportation process (DMTP),
which describes the transport processes, covering
the activities of all involved departments and all
levels of management;
    \item  improving the quality of information in the TPMS;
    \item formation of services for operational information
and technological interaction of participants in the
transportation process within the framework of a
single long-term, medium-term, shift-daily and current planning, execution and control of agreed and
approved plans;
    \item implementation of adaptive automatic control of
technological processes for operational work and
control over the execution of control decisions (CD);
    \item operational step-by-step and process assessment of
CD.
\end{itemize}

 The functioning of the ITCS is aimed at improving the
efficiency of the TP by:
\begin{itemize}[noitemsep]
    \item increasing the speed of traffic flows;
    \item reducing the turnaround time of the wagon, including by reducing the time spent by the wagon in an
empty state;
    \item reducing operating costs, including by increasings
the productivity of locomotives in freight traffic,
increasing the efficiency of using the car fleet;
    \item implementation of rational options for passing train
flows in a changing operational environment;
    \item reducing the number of overtimes at technical stations during the turnover of a freight car and increasing the transit capacity of car traffic.
    
\end{itemize}


 Currently, the following functional modules of the
ITCS have been implemented or are being implemented
at the Belarusian Railways. The AS ”Graphist” software
package (“Fig. 2”) allows you to develop train schedules
(DTS) [3], [4]. Currently, it is in commercial operation
at the Transportation Control Center of the Belarusian
Railways.
\end{multicols}


 \begin{figure}
     \centering
     \includegraphics[width=0.5\linewidth]{kek.png}
     
    Figure 1. Graphical interpretation of the methodology for the formation of ITCS.
     \end{figure}
     
          \begin{figure}
     \centering
     \includegraphics[width=0.75\linewidth]{изображение_2023-11-14_224433270.png}
     
         Figure 2. Modules for automatic construction of the DTS and its adjustment in AS ”Graphist”
         \end{figure}
 

     
\newpage
\begin{multicols}{2}


 The intelligent algorithm for the development of the
DTS is designed in such a way that, depending on the
relative position of trains in the DTS and their categories,
determine which of the station intervals should be used
in each specific case. The calculation is made taking
into account the mutual influence of the stations of the
section. The solution of the problem is envisaged at the
range of any length and configuration. Intellectualization
of the functions of the development of the DTS allowed
to reduce the workload of engineering personnel by 20-
30%. The introduction of AS ”Graphist” made it possible
to increase the sectional speed by 7-11.5% in some
sections and reduce the specific energy costs for train
traction by 3-6%.

 An automated system for shift-daily scheduling of
cargo work (AS SDS) has been developed and put into
commercial operation, which for the first time in the
world provides end-to-end scheduling of railway freight
work for the entire polygon of the road, all levels of
management (road, departmental, linear) and all planning
periods [5].

 The AS SDS (“Fig. 3”) implements the functions
of intelligently linking wagons to requests (taking into
account their condition, location, expiration category,
owner and other features), as well as other elements of
intelligent technologies: forecasting the time of arrival of
wagons at the station, adjusting planned indicators for the
second shift depending on their implementation for the
first; formation of plans taking into account the directive
establishment of an increased task for loading, etc.

 Based on the results of the operation of the AS SDS,
it was found that by improving the accuracy of planning,
the share of unscheduled loading decreased by 20-30%.
For the first time, a system of number-wise planning of
cargo work with high planning accuracy (91-94%) has
been implemented. The use of intelligent technologies in
the planning system made it possible to increase the ratio
of double operations by 8-12%; reduce the downtime of
a local car at individual freight stations of the station by
6-9%.

 An automated system for linking train formation with
a train schedule (LTFDTS) has been developed and put
into commercial operation (“Fig. 4”) [6].

 The main output decisions are: the schedule for the
departure of freight trains from train stations for the
forecast period; a plan for processing trains at train
stations during the forecast period; dislocation of trains
and wagons at train stations at the end of the forecast
period. An intelligent solution of LTFDTS is also an
abbreviated predictive DTS, in which, by means of
multifactorial selection, all trains participating in train
formation are linked to the threads of the predictive DTS.

 Intellectualization of the train formation planning process at the Belarusian Railway range using LTFDTS
made it possible to increase the efficiency of dispatching
control by increasing the reliability and automating the
development of a predictive train schedule with further
use in the automated train traffic control system (autodispatcher). This made it possible to enlarge the ranges
of train traffic control by 1.3-1.5 times and optimize
costs; reduce the time spent by trains and locomotives
at technical stations by reducing the waiting time of
technological operations from 15 to 20 percent; to ensure
the coordination of the predictive DTS with the train and
locomotive model of the road, reducing non-production
losses of locomotive crews up to 20 percent.

 Various development companies participated in the
creation of these and a number of other systems. One key
condition for their creation was to ensure the exchange
of CD between different systems of the ITCS. For these
purposes, a unified ontology of the transportation process
was formalized.
 The ontology of the transportation process presupposes
the existence of unified ways of describing the system
and the processes occurring in it. This task is inextricably
linked with the formation of a digital model of the
transportation process (DMTP). Actual mechanisms for
the formation of the CMPP should allow for the realtime simulation of the state of the TP. This requires the
unification of requirements for the content and form of
presentation of information about the parameters of the
functioning of objects [7] 

 DMTP may include:
 \begin{enumerate}
     \item models of objects (including resources) of the
TPMS; \item process models – a description of the processes
occurring both in the TPMS and in the external
object environment;
    \item models of the external object environment, describing the external impact on the objects of the
transportation process;
    \item situation forecasting models – the study of options
for the development of the transport situation in
case of emergency changes in the state of the
elements of the transport system, the external environment, with changes in the characteristics of
information flows;
    \item CD formation models that provide an analysis of
the operational environment and the formation of
effective CD;
    \item assessment models that provide an assessment of
the effectiveness of the implemented CD, the state
of objects and the parameters of the software.\
 \end{enumerate}
 
CMPP is focused on the implementation of dual control, i.e. adaptive control, in which not only the goals are
achieved, but also the model is refined.

 All DMTP objects are divided into the following subgroups:
 \begin{itemize}
    \item static objects (infrastructure objects);
    \item dynamic objects (tracking objects) (“Tab. I”).
\end{itemize}



\end{multicols}

\end{document}
