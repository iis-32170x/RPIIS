\documentclass[twocolumn]{article}
\usepackage[utf8]{inputenc}
\usepackage[left=16mm, top=18mm, right=16mm, bottom=18mm]
{geometry}
\usepackage{enumitem}
\usepackage{setspace}
\usepackage{indentfirst}
\usepackage{titlesec}
\usepackage{multicol}
\usepackage[english,russian]{babel}
\setlist[itemize]{itemsep=0pt, parsep=0pt, partopsep=0pt, topsep=0pt}
\setcounter{page}{255}
\titleformat{\section}{\large\centering\sc}{\thesection. }{0cm}{}[]
\begin{document}
 \noindent method can reach 85.58\% on the validation set and 85.00\% on the testing set. On the other hand, the average accuracy of classification on skeletons extracted by ZSM and OPCA is higher than the average accuracy of classification on skeletons extracted by ZS and OPTA. 
\par In addition, from the perspective of the classifiers, the accuracy of the classification of the decision tree and ensemble learning surpassed all the other classifiers. Based on the skeleton extracted by the MOPCA+DCEM+ATFM method, the decision tree and ensemble learning classifier can obtain 91.10\% and 92.80\% accuracy on the testing set, respectively. In contrast, the classification accuracy of other classifiers can only achieve about 85\%. Although the ensemble learning classifier has a slight bit advantage over the decision tree regarding classification accuracy, the time spent on training the ensemble learning is much more than the decision tree. 
\par In a word, for ten classes classification task, the best combination method is using MOPCA skeletonization to extract the skeleton, using ATFM and DCEM to offset the noise’s influence, and using ensemble learning to predict the class of the static hand gesture. The overall accuracy can reach 91.1\%, and the train time is 0.6134s.
\begin{center}
V. CONCLUSION
\end{center}
\par The hand gesture recognition based on the new image skeletonization methods, extracted gesture feature vector and using machine learning technique allow us to increase the classification accuracy. For 5 classes and 10 classes hand gesture classification task, the improvement of accuracy on test set is within the range of 0.4\% to 20.4\% , and that of 5\% to 18\% . The MOPCA+ADFM+DCEM method is effective in terms of average classification accuracy on test set. It achieves 97.5\% on 5 classes recognition task and 85.00\% on 10 classes recognition task. In addition, for 5 classes recognition task and 10 classed recognition task , the training time consumed by six classifiers is within the range of 0.7s to 8.9s and that of the 0.3s to 11s, respectively. It is set that ensemble learning model is the best classifier and it allows us to achieve 100\% (5 classes) and 92.8\% (10 classes) on test set. Increasing the accuracy of hand gesture classification based on the proposed skeletonization methods improves the technical characteristics of intelligent systems using video interfaces for entering commands and data, and makes a significant contribution to the development of semantic technologies for designing such systems.
\begin{center}
    REFERENCES
\end{center}
\begin{itemize}
 \footnotesize{
  \renewcommand{\labelitemi}{[1]}
  \item  S. Sharma and S. Singh, “Vision-based hand gesture recognition using deep learning for the interpretation of sign language,”
  \textit {Expert Systems with Applications}, vol. 182, p. 115657, 2021. 
 \renewcommand{\labelitemi}{[2]}
  \item J. Zeng, Y. Sun, and F. Wang, “A natural hand gesture system for intelligent human-computer interaction and medical assistance,” in 2012 \textit{Third Global Congress on Intelligent Systems.} IEEE, 2012, pp. 382–385. 
  \renewcommand{\labelitemi}{[3]}
  \item M. Van den Bergh, D. Carton, R. De Nijs, N. Mitsou, C. Landsiedel, K. Kuehnlenz, D. Wollherr, L. Van Gool, and M. Buss, “Real-time 3d hand gesture interaction with a robot for understanding directions from humans,” in \textit{2011 Ro-Man.} IEEE, 2011, pp. 357–362. 
  \renewcommand{\labelitemi}{[4]}
  \item M. Oudah, A. Al-Naji, and J. Chahl, “Hand gesture recognition based on computer vision: a review of techniques,”\textit{ journal of Imaging,} vol. 6, no. 8, p. 73, 2020. 
  \renewcommand{\labelitemi}{[5]}
  \item L. Dipietro, A. M. Sabatini, and P. Dario, “A survey of glovebased systems and their applications,” \textit{Ieee transactions on systems, man, and cybernetics, part c (applications and reviews),} vol. 38, no. 4, pp. 461–482, 2008. 
  \renewcommand{\labelitemi}{[6]}
  \item J. J. LaViola Jr, “A survey of hand posture and gesture recognition techniques and technology,” 1999. 
  \renewcommand{\labelitemi}{[7]}
  \item G. Murthy and R. Jadon, “A review of vision based hand gestures recognition,” \textit {International Journal of Information Technology and Knowledge Management,} vol. 2, no. 2, pp. 405–410, 2009. 
  \renewcommand{\labelitemi}{[8]}
  \item H. Kaur and J. Rani, “A review: Study of various techniques of hand gesture recognition,” in \textit{2016 IEEE 1st international conference on power electronics, intelligent control and energy systems (ICPEICES).} IEEE, 2016, pp. 1–5. 
  \renewcommand{\labelitemi}{[9]}
  \item D. Konstantinidis, K. Dimitropoulos, and P. Daras, “Sign language recognition based on hand and body skeletal data,” in \textit{2018-3DTV-Conference: The True Vision-Capture, Transmission and Display of 3D Video (3DTV-CON).} IEEE, 2018, pp. 1–4. 
  \renewcommand{\labelitemi}{[10]}
  \item C. Xi, J. Chen, C. Zhao, Q. Pei, and L. Liu, “Real-time hand tracking using kinect,” in \textit{Proceedings of the 2nd International Conference on Digital Signal Processing,} 2018, pp. 37–42. 
  \renewcommand{\labelitemi}{[11]}
  \item G. Devineau, F. Moutarde, W. Xi, and J. Yang, “Deep learning for hand gesture recognition on skeletal data,” in \textit{2018 13th IEEE International Conference on Automatic Face \& Gesture Recognition (FG 2018).} IEEE, 2018, pp. 106–113. 
  \renewcommand{\labelitemi}{[12]}
  \item F. Jiang, S. Wu, G. Yang, D. Zhao, and S. Kung, “independent hand gesture recognition with kinect,”\textit{ Signal, Image and Video Processing,} vol. 8, pp. 163–172, 2014. 
  \renewcommand{\labelitemi}{[13]}
  \item F. Y. Shih and W.-T. Wong, “Fully parallel thinning with tolerance to boundary noise,” \textit{Pattern Recognition,} vol. 27, no. 12, pp. 1677–1695, 1994. 
  \renewcommand{\labelitemi}{[14]}
  \item J. Feldman and M. Singh, “Bayesian estimation of the shape skeleton,” \textit{Proceedings of the National Academy of Sciences,} vol. 103, no. 47, pp. 18 014–18 019, 2006. 
  \renewcommand{\labelitemi}{[15]}
  \item F. Gao, G. Wei, S. Xin, S. Gao, and Y. Zhou, “2d skeleton extraction based on heat equation,” \textit{Computers and Graphics,} vol. 74, pp. 99–108, 2018. 
  \renewcommand{\labelitemi}{[16]}
  \item C. Yang, B. Indurkhya, J. See, and M. Grzegorzek, “Towards automatic skeleton extraction with skeleton grafting,” \textit{IEEE transactions on visualization and computer graphics,} vol. 27, no. 12, pp. 4520–4532, 2020. 
  \renewcommand{\labelitemi}{[17]}
  \item W. Shen, X. Bai, R. Hu, H. Wang, and L. J. Latecki, “Skeleton growing and pruning with bending potential ratio,” \textit{Pattern Recognition,} vol. 44, no. 2, pp. 196–209, 2011. 
  \renewcommand{\labelitemi}{[18]}
  \item A. S. Montero and J. Lang, “Skeleton pruning by contour approximation and the integer medial axis transform,” \textit{Computers and Graphics,} vol. 36, no. 5, pp. 477–487, 2012. \renewcommand{\labelitemi}{[19]}
  \item L. Serino and G. S. di Baja, “A new strategy for skeleton pruning,” \textit{Pattern Recognition Letters,} vol. 76, pp. 41–48, 2016. 
  \renewcommand{\labelitemi}{[20]}
  \item M. E. Hoffman and E. K. Wong, “Scale-space approach to image thinning using the most prominent ridge-line in the image pyramid data structure,” in \textit{Document Recognition V,} vol. 3305. SPIE, 1998, pp. 242–252. 
  \renewcommand{\labelitemi}{[21]}
  \item J. Cai, “Robust filtering-based thinning algorithm for pattern recognition,” \textit{The Computer Journal,} vol. 55, no. 7, pp. 887–896, 2012. 
  \renewcommand{\labelitemi}{[22]}
  \item R. T. Chin, H.-K. Wan, D. Stover, and R. Iverson, “A one-pass thinning algorithm and its parallel implementation,” \textit{Computer Vision, Graphics, and Image Processing,} vol. 40, no. 1, pp. 30– 40, 1987. 
  \renewcommand{\labelitemi}{[23]}
  \item T. Y. Zhang and C. Y. Suen, “A fast parallel algorithm for thinning digital patterns,” \textit{Communications of the ACM,} vol. 27, no. 3, pp. 236–239, 1984. 
  \renewcommand{\labelitemi}{[24]}
  \item J. Ma, T. V. Yurevich, and V. K. Kanapelka, “Image skeletonization based on combination of one- and two-sub-iterations models(in russ.),” \textit{Informatics,} vol. 17, no. 2, pp. 25–35, 2020.} 
  \renewcommand{\labelitemi}{[25]}
  \item J. Ma, X.-H. Ren, T. V. Yurevich, and V. K. Kanapelka, “A
novel sub-iterative parallel skeletonization method,” \textit{Journal of Computers} (Taiwan), vol. 32, no. 6, pp. 83–97, 2021.
\renewcommand{\labelitemi}{[26]}
  \item J. Ma, X. Ren, V. Y. Tsviatkou, and V. K. Kanapelka, “A novel
fully parallel skeletonization algorithm,” \textit{Pattern Analysis and
Applications,} pp. 1–20, 2022.
\renewcommand{\labelitemi}{[27]}
  \item J. Ma, X. Ren, V. K. Kanapelka, and V. Y. Tsviatkou, “An
automatic pruning method for skeleton images,” in \textit{Proceedings
of the 15th International Conference,2021.} United Institute of
Informatics Problems of the National Academy of Sciences of
Belarus, 2021, pp. 232–235.
\renewcommand{\labelitemi}{[28]}
  \item X. Bai, L. J. Latecki, and W.-Y. Liu, “Skeleton pruning by contour
partitioning with discrete curve evolution,” \textit{IEEE transactions on
pattern analysis and machine intelligence,} vol. 29, no. 3, pp. 449–
462, 2007.
\renewcommand{\labelitemi}{[29]}
  \item J. Ma, X. Ren, H. Li, W. Li, V. Y. Tsviatkou, and A. A.
Boriskevich, “Noise-against skeleton extraction framework and
application on hand gesture recognition,” \textit{IEEE Access,} vol. 11,
pp. 9547–9559, 2023.
\renewcommand{\labelitemi}{[30]}
  \item H. Chatbri and K. Kameyama, “Using scale space filtering to
make thinning algorithms robust against noise in sketch images,”
\textit{Pattern Recognition Letters,} vol. 42, pp. 1–10, 2014.
\renewcommand{\labelitemi}{[31]}
  \item O. Z. Maimon and L. Rokach, \textit{Data mining with decision trees:
theory and applications.} World scientific, 2014, vol. 81.
\renewcommand{\labelitemi}{[32]}
  \item T. Cover and P. Hart, “Nearest neighbor pattern classification,”
\textit{IEEE transactions on information theory,} vol. 13, no. 1, pp. 21–
27, 1967.
\renewcommand{\labelitemi}{[33]}
  \item D. J. Hand and K. Yu, “Idiot’s bayes—not so stupid after all?”
\textit{International statistical review,} vol. 69, no. 3, pp. 385–398, 2001.
\renewcommand{\labelitemi}{[34]}
  \item C. Cortes and V. Vapnik, “Support-vector networks,” \textit{Machine
learning,} vol. 20, pp. 273–297, 1995.
\renewcommand{\labelitemi}{[35]}
  \item T. K. Ho, “Random decision forests,” in \textit{Proceedings of 3rd
international conference on document analysis and recognition,}
vol. 1. IEEE, 1995, pp. 278–282.
\renewcommand{\labelitemi}{[36]}
  \item T. Hastie, R. Tibshirani, J. H. Friedman, and J. H. Friedman,
\textit{The elements of statistical learning: data mining, inference, and
prediction.} Springer, 2009, vol. 2.
\end{itemize}
\newpage
\begin{center}
\large
\textbf{Распознавание жестов рук на основе
свойств скелетизированных изображений} \\
Ма Ц., Цветков В. Ю., Борискевич А. А.
\end{center}
\normalsize
\par Распознавание жестов рук является важной задачей и может использоваться во многих практических приложениях. В интеллектуальных системах
распознавание жестов рук может использоваться для
ввода информации посредством видеоинтерфейса. В
настоящее время распознавание жестов рук на основе скелета стало популярной темой исследований.
Существующие методы имеют низкую дискриминационную способность из-за чувствительности признаков
к шуму изображения. Мы предложили новые методы
уменьшения влияния шума на выделение признаков
изображения руки. Разработан новый метод распознавания жестов рук, основанный на свойствах скелетизированных изображений. Цель исследования состоит
в повышении точности классификации жестов рук.
Данный подход позволяет повысить точность классификации с 5\%
до 21\%
по сравнению с существующими
известными методами.\\
\begin{flushright}
Received 12.03.2023
\end{flushright}
\newpage
\onecolumn
\begin{center}
\Huge
\setstretch{2.5}
\textbf{Automation of Educational Activities within
the OSTIS Ecosystem}
\singlespacing
\end{center}
\begin{multicols}{2}
\raggedright
\Large
 \hspace{0.5cm} Alena Kazlova and Aliaksandr Halavaty\\
 \hspace{2cm} \textit{Belarusian State University}\\
 \hspace{3.5cm}  Minsk, Belarus\\
  \hspace{2.5cm} Email: kozlova@bsu.by,\\
   \hspace{2.2cm} alex.halavatyi@gmail.com
\columnbreak

\raggedleft
\Large
 Natalia Grakova  \ \ \ \ \ \ \ \ \ \ \ \ \ \ \ \ \ \ \ \ \\ 
 \textit{Belarusian State University of}\ \ \ \ \ \  \ \ \ \ \\
 \textit{Informatics and Radioelectronics}\ \ \ \ \ \ \ \ \ \\
 Minsk, Belarus \ \ \ \ \ \ \ \ \ \ \ \ \ \ \ \ \ \ \  \\ 
 grakova@bsuir.by \ \ \ \ \ \ \ \ \ \ \ \ \ \ \ \
\end{multicols}
\begin{multicols}{2}
\setstretch{0.8}
 \textbf{\textit{Abstract}—An analysis of the need for a comprehensive
restructuring of the education system, taking into account
the requirements of the digital economy, is presented. The
ways of solving some problems of the implementation
of the educational process at the level of general school
education are determined. A semantic approach to building
a complex of intelligent learning subsystems including
teaching, assisting learning and analytical to accompany the
learning process ones within the framework of the OSTIS
ecosystem is proposed.}
 \par 
  \textbf{\textit{Keywords}—intelligent learning systems, semantics, OSTIS ecosystem, knowledge processing}
  \normalsize
  \setstretch{1}\\
\begin{center}
I. INTRODUCTION
\end{center}
\par In the context of the transition to the information
society and the comprehensive digitalization of all areas
of human activity at its various levels, highly qualified
personnel are of the greatest value. The volume and
level of requirements for the presentation and use of
information in all spheres of life is increasing, which
entails the inevitable active involvement of professionals
in the process of continuous education. A modern person
in the information society must be able to adapt to rapidly
changing information flows. The formation of such skills
is the main task of every educational institution, including universities, which in modern conditions are subject
to increasingly stringent requirements. This applies to
both the level of teaching and the level of organization of educational activities. Today, the organization
of educational activities in schools and in secondary
specialized vocational, higher educational institutions
largely determines the level of development of the state.
Therefore, we can fully explain the great interest in the
use of information technology in order to increase the
efficiency of this activity. However, despite the rather
active research carried out in this direction, it is too
early to say that the use of information technology has
significantly increased the effectiveness of educational
activities. There are many reasons for this. Among
them there are both objective technical, methodological
reasons, as well as reasons of a purely organizational,
administrative nature. These reasons include:
\begin{itemize}
\item lack of a systematic approach to the selection of the
main objects and processes for automating the activities of educational institutions, including higher
ones;
\item a small number of viable technologies for the
integrated development, implementation, operation,
maintenance and evolution of educational automation tools;
\item natural social resistance, conservatism that impedes
the comprehensive automation of the educational
process.
\end{itemize}
\begin{center} 
II. CURRENT APPROACHES TO THE EDUCATIONAL ACTIVITIES AUTOMATION
\end{center}
\par Automation of educational activities requires an integrated approach, taking into account the peculiarities
of educational work at all stages of education, from
elementary school to graduation from the magistracy,
and possibly further, when obtaining higher qualifications
in graduate school. This approach certainly requires the
use of existing ones, as well as further development and
widespread use of methods and tools of artificial intelligence, and related disciplines. Work in this direction
has been carried out by various groups of researchers
for more than 30 years. Research is being carried out on
the theory and methodology of distance education, new
approaches to the development of distance and open education are proposed, based on the ideas of organizational
design and reengineering of organizations, methods of
knowledge engineering and the theory of agents, models
of multi-agent systems and virtual organizations [1]--[7].
\par A number of authors identified and studied the main
classes of systems and technologies needed to create
virtual departments of universities and universities in
general. Particular attention of researchers is paid to
the problems of symbiosis of network and intelligent
technologies, for example, models of intelligent learning
systems based on multi-agent technologies [8], [9].
\par One of the current trends in the development of applied
intelligent systems (IS) is the implementation of ISs
that can not only solve problems from the relevant
\end{multicols}
\end{document}
