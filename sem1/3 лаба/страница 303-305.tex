\documentclass{article}
\usepackage[utf8]{inputenc}
\usepackage[left=17mm, top=17mm, right=17mm, bottom=5mm, nohead, nofoot]{geometry}
\usepackage{enumitem}
\usepackage{multicol}
\usepackage{titlesec}
\usepackage{titlesec}
\usepackage[russian]{babel}
\setlist[itemize]{itemsep=0pt, parsep=0pt, partopsep=0pt, topsep=0pt}
\setcounter{page}{303}
\titleformat{\section}{\large\centering\sc}{\thesection. }{0cm}{}[]
\begin{document}
\begin{multicols}{2}

\begin{tabbing}
	\hspace{-0.17cm} localization type into:\textit{ area objects\^\,\ linear (multilinear)}\\ \ \textit{objects\^\ }, and \textit{point objects\^\ } .
\end{tabbing}
          At the next stage of developing the ontology of
\textit{terrain objects}, we will set the subdivision of \textit{terrain
objects} on orthogonal bases, which corresponds to the
placement of objects in accordance with thematic layers
in \textit{geoinformation systems}.
\ \\

For each \textit{terrain object}, the main semantic characteristics inherent only 
to it are highlighted. It should be
particularly noted that metric characteristics do not have
such a property. According to this classifier, each class
of \textit{terrain objects} has a unique unambiguous designation.
The classifier hierarchy has eight classification stages and
consists \textit{of the class code, subclass code, group code,
subgroup code, order code, suborder code, species code,
subspecies code}. Thus, thanks to the coding method,
generic relations have already been defined, reflecting
the correlation of various \textit{terrain object classes}, and the
characteristics of a specific \textit{terrain object class} have also
been established. Due to the fact that the basic properties
and relations are set not of specific \textit{physical objects} but of
their classes, then such information is meta-information in
relation to specific \textit{terrain objects}, and the totality of this
meta-information is an ontology of \textit{terrain objects}, which
in turn is part of the \textit{knowledge base} of the \textit{intelligent
geoinformation system}.
\begin{tabbing}
\hspace{0cm}\textbf{terrain object}\\
 \(\Rightarrow\) \ \ \ \ \ \   \textit{subdividing*:}\\
 \ \ \ \ \ \ \ \ \ \ \ \textbf{Typology of terrain objects by localization\^\\ \ \ \ \ \ \ \ \ \ \ \  =  \{ }
\end{tabbing}
\begin{adjustwidth}{15mm}{}
 \begin{itemize}  
 •  \ \ \ \     point terrain object
    \begin{itemize}
        \ \(\Rightarrow\) inclusion*:\\
        \begin{itemize}
     \ \ \ \ • well\\
       \ \ \ \      • light post
        \end{itemize}
    \end{itemize}
    • \ \ \ \ linear terrain object
    \begin{itemize}
         \(\Rightarrow\)inclusion*:\\
        \begin{itemize}
            \ \ \ \ • bridge
        \end{itemize}
    \end{itemize}
    • \ \ \ \ multilinear terrain object
    \begin{itemize}
         \(\Rightarrow\)inclusion*:\\
        \begin{itemize}
        \ \ \ \  • river\\
        \ \ \ \ • road
        \end{itemize}
    \end{itemize}
    • \ \ \ \ area terrain object
    \begin{itemize}
         \(\Rightarrow\)inclusion*:\\
        \begin{itemize}
           \ \ \ \  • lake\\
          \ \ \ \  • administrative area
        \end{itemize}
    \end{itemize}
\end{itemize}
\ \ \ \ \ \}
\end{adjustwidth}
\begin{center}
 VII. SPECIFICATION OF THE MAP LANGUAGE
 \end{center}
\ \ \ The \textit{Map Language} belongs to the family of semantic
compatible languages – \textit{sc-languages} – and is intended for
the formal description of \textit{terrain objects} and the relations
between them in \textit{geoinformation systems}. Therefore, the
\textbf{Map Language Syntax}, like \textit{syntax} of any other \textit{sclanguage}, is the \textit{Syntax} of the \textit{SC-code}. This approach
allows:\\
\ \\ \ \\ \ 
	\begin{itemize}
		\item using a minimum of means to interpret the specified
\textit{terrain objects} on the map;

		\item using the \textit{Question Language for ostis-systems};
	
		\item  reducing the search to most of the given \textit{questions}
to searching for information in the current state of
the \textit{ostis-system knowledge base}
	\end{itemize}
\ \\  

  \textbf{Denotational semantics of the Map Language} includes
the \textit{Subject domain and the ontology of terrain objects}
and \textit{their geosemantic elements}.
\begin{center}
VIII. AUTOMATION TOOLS FOR THE INTELLIGENT
GEOINFORMATION SYSTEMS DESIGN
\end{center}
The design of intelligent geoinformation systems is
carried out in stages. At the first stage, the knowledge
base of the subject domain is formed and for this purpose
an electronic map (voluntary cartographic information)
is analyzed and translated into the knowledge base of
terrain objects with the establishment of geosemantic
elements for the corresponding territory. At this stage, it
is determined, firstly, to which class the terrain object
under study belongs and, further, depending on the type
of object, the concept of a knowledge base corresponding
to a specific physical terrain object is formed. Thus, many
concepts are created that describe specific terrain objects
for each class of terrain objects. It should be noted that
it is at this stage of the formation of the knowledge base
that semantic elements are established. At the second
stage of designing an intelligent geoinformation system,
the knowledge base obtained at the first stage is integrated
with external knowledge bases. At this stage, in addition
to geographical knowledge, knowledge of related subject
domains is added, thereby it becomes possible to establish
interdisciplinary connections. An illustrative example is
integration with biological classifiers, which in implementation represent an ontology of flora and fauna objects.
Such integration expands the functional and intelligent
capabilities of the applied intelligent geoinformation
system. Note that at this stage, homonymy is removed in
the names of geographical objects belonging to the classes
of settlements. For settlements of the Republic of Belarus,
this is achieved by using the \textit{system of designations of
administrative-territorial division objects and settlements}
and semantic comparison of geographical terrain objects
is carried out according to the following principle:\\
\begin{itemize}
  \item the terrain object class is determined;
  \item the terrain object subclass, species, subspecies, etc.
is determined in accordance with the classifier of
terrain objects, i.e. types of terrain objects in the
ontology;
  \item the attributes and characteristics that are inherent in
this terrain object class are determined;
  \item the values of the characteristics for this object class
are determined;
  \item the homonymy of identification is eliminated;
 \end{itemize}	
\newpage
\begin{itemize}
  \item appropriate connections are established between the
map object, the concept in the knowledge base with
the established geosemantic elements;
  \item spatial relations are established between terrain
objects assigned to certain classes.
 \end{itemize}\\
 
\begin{center}
 CONCLUSION
 \end{center}\\
 
\ \ \ \ Let us list the main provisions of this article:
\begin{itemize}
  \item  the development of geoinformation systems consists
in their intellectualization, thereby expanding the
range of applied problems using knowledge about
terrain objects;
  \item it is proposed to consider the map as an \textit{information
construction}, the elements of which are \textit{terrain
objects}, thereby ensuring the structural and semantic
interoperability of geoinformation systems due to the
transition from the map to the semantic description of
map elements, that is, terrain objects and connections
(spatial relations) between them;
  \item  ensuring semantic interoperability is achieved
through the development of ontologies of subject
domains, and the establishment of  \textit{geosemantic
elements} allows setting spatial characteristics of
terrain objects;
  \item availability of a particular  \textit{Technology for intelligent
geoinformation systems design} provides the process
of designing intelligent geoinformation systems built
on the principles of ostis-systems.
 \end{itemize}
\begin{center}
 ACKNOWLEDGMENT
 \end{center}
\ \ \ \  \ The author would like to thank the research group of
the Department of Intelligent Information Technologies
of the Belarusian State University of Informatics and
Radioelectronics for its help in the work and valuable
comments.
\begin{center}
 REFERENCES
 \end{center}
\begin{itemize}
 \footnotesize{
  \renewcommand{\labelitemi}{[1]}
		\item A. Kryuchkov, S. Samodumkin, M. Stepanova, and N. Gulyakina,
 \textit{Intellektual’nye tekhnologii v geoinformacionnyh sistemah [Intelligent technologies in geoinformation systems]}. BSUIR, 2006, p.
202 (In Russ.).
\renewcommand{\labelitemi}{[2]}
		\item S. Ablameyko, G. Aparin, and A. Kryuchkov,  \textit{Geograficheskie
informacionnye sistemy. Sozdanie cifrovyh kart [Geographical
information systems. Creating digital maps]}. Institute of Technical
Cybernetics of the National Academy of Sciences of Belarus, 2000,
p. 464 (In Russ.).
\renewcommand{\labelitemi}{[3]}	
\item  Ya. Ivakin, “Metody intellektualizacii promyshlennyh
geoinformacionnyh sistem na osnove ontologij [Methods
of intellectualization of industrial geoinformation systems based
on ontologies] ,” Doct. diss.: 05.13.06, Saint-Petersburg, 2009,
(In Russ.).
\renewcommand{\labelitemi}{[4]}
\item M. Belyakova,  \textit{Intellektual’nye geoinformacionnye sistemy dlya
upravleniya infrastrukturoj transportnyh kompleksov [Intelligent
geoinformation systems for infrastructure management of transport
complexes]}. Taganrog : Southern Federal University Press, 2016,
p. 190 (In Russ.).
\renewcommand{\labelitemi}{[5]}
\item A. Gubarevich, O. Morosin, and D. Lande, “Ontologicheskoe
proektirovanie intellektual’nyh sistem v oblasti istorii [Ontological
design of intelligent systems in the field of history],”  \textit{Otkrytye
semanticheskie tekhnologii proektirovaniya intellektual’nyh sistem
[Open semantic technologies for designing intelligent systems]},
pp. 245–250, 2017, (In Russ.).
\renewcommand{\labelitemi}{[6]}
\item A. Gubarevich, S. Vityaz, and R. Grigyanets, “Struktura baz znanij
v intellektual’nyh sistemah po istorii [Structure of knowledge
bases in intelligent systems on history],”  \textit{Otkrytye semanticheskie
tekhnologii proektirovaniya intellektual’nyh sistem [Open semantic
technologies for designing intelligent systems]}, pp. 347–350, 2018,
(In Russ.).
\renewcommand{\labelitemi}{[7]}
\item A. Bliskavitsky,  \textit{Konceptual’noe proektirovanie GIS i upravlenie
geoinformaciej. Tekhnologii integracii, kartograficheskogo
predstavleniya, veb-poiska i rasprostraneniya geoinformacii
[Conceptual GIS design and geoinformation management.
Technologies of integration, cartographic representation, web
search, and distribution of geoinformation]}. \textbf{LAP LAMBERT}
Academic Publishing, 2012, p. 484 (In Russ.).
\renewcommand{\labelitemi}{[8]}
\item  ——,“Semantika geoprostranstvennyh ob"ektov, funkcional’naya
grammatika i intellektual’nye GIS [Semantics of geospatial objects,
functional grammar, and intelligent GIS],” in  \textit{Izvestiya vysshih
uchebnyh zavedenij. Geologiya i razvedka [News of higher
educational institutions. Geology and exploration]}, no. 2, 2014,
pp. 62–69, (In Russ.).
\renewcommand{\labelitemi}{[9]}
\item  Y. Hu, “Geospatial semantics,”  \textit{Comprehensive Geographic Information Systems}, pp. 80–94, 2018.
\renewcommand{\labelitemi}{[10]}
\item K. Janowicz, S. Simon, T. Pehle, and G. Hart, “Geospatial
semantics and linked spatiotemporal data — past, present, and
future,”  \textit{Semantic Web}, vol. 3, pp. 321–332, 10 2012.
\renewcommand{\labelitemi}{[11]}
\item S. Samodumkin, “Next-generation intelligent geoinformation
systems,”  \textit{Otkrytye semanticheskie tekhnologii proektirovaniya intellektual’nykh system [Open semantic technologies for intelligent systems]}, 2022.
\renewcommand{\labelitemi}{[12]}
\item  \textit{Cifrovye karty mestnosti informaciya, otobrazhaemaya na
topograficheskih kartah i planah naselennyh punktov : OKRB
012-2007 [Digital maps of the area information displayed on
topographic maps and plans of settlements} : NKRB 012–2007],
Minsk, 2007, (In Russ.).  
\end{itemize}\\
\
\

\begin{center}
 \textbf{\Large{Поддержка жизненного цикла
интеллектуальных геоинформационных
систем различного назначения }}
 \end{center}
 \begin{center}
\Large{Самодумкин С.А}\\

 \end{center}
 Работа посвящена частной технологии проектирования интеллектуальных геоинформационных систем,
построенных по принципам ostis-систем. Структурная и
семантическая интероперабельность геоинформационных систем, построенных по предлагаемой технологии,
обеспечивается за счет перехода от карты к семантическому описанию элементов карты.

\ \ \ \ \ \ \ \ \ \ \ \ \ \ \ \ \ \ \ \ \ \ \ \ \ \ \ \ \ \ \ \ \ \ \ \ \ \ \ \ \ \ \ \ \ \ Received 01.03.2023 \ \\ \ \\ \ \\ \ \\ \ \\ \ \\ \ \\ \ \\ \ \\ \ \\ \ \\ 
\end{multicols}
\newpage
\ \\
\begin{center}
\LARGE\textbf{Ensuring Information Security of the OSTIS
Ecosystem}
\end{center}\ \\  
\begin{multicols}{2}
\raggedright
\Large
 \hspace{4cm} Valery Chertkov\\
 \hspace{2.5cm} \textit{Euphrosyne Polotskaya State}\\
 \hspace{3cm}  \textit{University of Polotsk}\\
  \hspace{4cm} Polotsk, Belarus\\
   \hspace{3.5cm} v.chertkov@psu.by
\columnbreak

\raggedleft
\Large
 Vladimir Zakharau   \ \ \ \ \ \ \ \ \ \ \\ 
\textit{Belarusian State University of}\\
 \hspace{-3cm}\textit{Informatics and Radioelectronics}\\
 \hspace{-4cm}Minsk, Belarus \ \ \ \ \ \ \ \ \ \ \\ 
 \hspace{-3.5cm}zakharau@bsuir.by \ \ \ \ \ \ \ \ \ \
\end{multicols}
\ \\ 
\begin{multicols}{2}

\textbf{
  \textit{Abstract}—The development of artificial intelligence systems, associated with the transition to working with
knowledge bases instead of data, requires the formation of
new approaches to ensuring information security systems.
The article is devoted to the review of approaches and
principles of ensuring security in intelligent systems of
the new generation. The current state of methods and
means of ensuring information security in intelligent systems
is considered and the main goals and directions for
the development of information security ostis-systems are
formed. The information security methods presented in the
article are extremely important when designing the ostissystems security system and analyzing their security level.
}\\

\textbf{ \textit{Keywords}—information security, new generation intelligent system, Information security threats
}
\begin{center}
 I. INTRODUCTION
 \end{center}\\

 A wide variety of information security models, the
growing amount of data that needs to be analyzed to
detect attacks on information systems, the variability of
attack methods and the dynamic change in protected
information systems, the need for a rapid response to
attacks, the fuzziness of the criteria for detecting attacks
and the choice of methods and means of responding to
them, the lack of highly qualified security specialists
entails the need to use artificial intelligence methods to
solve security problems.
\begin{center}
 II. THE SPECIFICS OF ENSURING INFORMATION
SECURITY OF INTELLIGENT SYSTEMS OF A NEW
GENERATION
 \end{center}
 
 Information security of intelligent systems should be
considered from two points of view:
\begin{itemize}[noitemsep]
    \item application of artificial intelligence in information
security;
    \item organization of information security in intelligent
systems.
\end{itemize}

\textbf{The use of artificial intelligence in information
security}

Artificial intelligence is actively used to monitor and
analyze security vulnerabilities in information transmission networks [1]. The artificial intelligence system allows
machines to perform tasks more efficiently, such as: 
\begin{itemize}[noitemsep]
    \item visual perception, speech recognition, decision making and translation from one language to another;
    \item invasion detection - artificial intelligence can detect
network attacks, malware infections and other cyber
threats;
systems.
     \item cyber analytics - artificial intelligence is also used
to analyze big data in order to identify patterns and
anomalies in the organization’s cyber security system
in order to detect not only known, but also unknown
threats;
    \item secure software development - artificial intelligence
can help create more secure software by providing
real-time feedback to developers.

\end{itemize}
Artificial intelligence is used not only for protection,
but also for attack, for example, to emulate acoustic,
video and other images in order to deceive authentication
mechanisms and further impersonation, deceive checking
a person or robot capcha, etc.

Currently, it is possible to define the following classes
of systems in which artificial intelligence is used [2]:
\begin{itemize}[noitemsep]
    \item UEBA (User and Entity Behavior Analytics) —
a system for analyzing the behavior of subjects
(users, programs, agents, etc.) in order to detect nonstandard behavior and use them to detect potential
threats using threat templates (patterns);

    \item IP (Threat Intelligence Platform) — platforms for
early detection of threats based on the collection
and analysis of information from indicators of
compromise and response to them. The use of
machine learning methods increases the efficiency
of detecting unknown threats at an early stage;

    \item EDR (Endpoint Detection and Response) — attack
detection systems for rapid response at the end
points of a computer network. Can detect malware,
autom
\end{itemize}
\end{multicols}
\end{document}
