
\documentclass{article}
\usepackage[utf8]{inputenc}
\usepackage[left=17mm, top=17mm, right=17mm, bottom=5mm, nohead, nofoot]{geometry}
\usepackage{enumitem}
\usepackage{multicol}
\usepackage{titlesec}
\usepackage{titlesec}
\usepackage[russian]{babel}
\setlist[itemize]{itemsep=0pt, parsep=0pt, partopsep=0pt, topsep=0pt}
\setcounter{page}{309}
\titleformat{\section}{\large\centering\sc}{\thesection. }{0cm}{}[]
\begin{document}
\begin{multicols}{2}


Active means of destroying the semantics of knowledge
bases (semantic viruses) should also be singled out as a
separate category of information security threats [6].

\textbf{Knowledge base access control policy}

Mandatory access control (MAC) is based on mandatory (forced) access control, which is determined by
four conditions: all subjects and objects of the system
are identified; a lattice of information security levels is
specified; each object of the system is assigned a security
level that determines the importance of the information
contained in it; each subject of the system is assigned an
access level that determines the level of trust in him in
the intellectual system. In addition, the mandate policy
has a higher degree of reliability. The implementation
of this policy is based on the developed algorithm for
determining the agreed security levels for all elements of
the ontology.
\par
Since semantic knowledge bases, unlike a relational
database, allow executing rules for obtaining logical
conclusions, it is relevant to ensure data security by
developing algorithms and methods that can only receive
data that have security levels less than the access levels
of the subjects who requested them [7].
\par
\textbf{Connectivity}
\par
All information stored in the semantic memory of
the intelligent system is systematized in the form of a
single knowledge base. Such information includes directly
processed knowledge, interpreted programs, formulations
of tasks to be solved, plans and protocols for solving problems, information about users, a description of the syntax
and semantics of external languages, a description of the
user interface, and much more [8]. In the information
knowledge base between fragments of information (units
of information), the possibility of establishing links of
various types should be provided. First of all, these links
can characterize the relationship between information
units. Violation of connections leads to an incorrect
logical conclusion, or to obtaining false knowledge, or
to incompatibility of knowledge in the base.
\par
\textbf{Introduction of semantic metric}
\par
On a set of information units, in some cases it is useful
to set a relation that characterizes the semantic proximity
of information units, i.e. the force of the associative
connection between information units [9]. It could be
called the relevance relation for information units. This
attitude makes it possible to single out some typical
situations in the knowledge base. The relevance relation
when working with information units allows you to find
knowledge that is close to what has already been found.
\par
\textbf{Semantic Compatibility}
\par
Internal semantic compatibility between the components of an intelligent computer system (i.e., the maximum
possible introduction of common, coinciding concepts for
various fragments of a stored knowledge base), which
is a form of convergence and deep integration within
an intelligent computer system for various types of
knowledge and various problem solving models, which
ensures effective implementation of the multimodality
of an intelligent computer system. External semantic
compatibility between various intelligent computer systems, which is expressed not only in the commonality of
the concepts used, but also in the commonality of basic
knowledge and is a necessary condition for ensuring a
high level of socialization of intelligent computer systems
[10].
\par
\textbf{Activity}\par
In an intellectual system, the knowledge available in
this system contributes to the actualization of certain
actions. Thus, the execution of activities in an intelligent
system should be initiated by the current state of the
knowledge base. The appearance in the database of facts
or descriptions of events, the establishment of links
can become a source of system activity [11]. Including
deliberate distortion of information and connections can
become a source of deliberate distortion of information.
\par
For new generation intelligent systems, there are a
number of aspects that require the development of new
algorithms and methods for ensuring information security
in addition to existing mechanisms:



  \begin{itemize}
    \item multi-level access to individual parts of the knowledge base, as information can be public, personal,
confidential;

    \item monitoring of changes in the meanings of words over
time, as well as the meanings of translation from a
foreign language that may influence decisions;

  
    \item  protection against unauthorized use by using cryptosemantic ciphers;
         \item constant monitoring of vulnerabilities in the system;
         \item  logging of actions (interactions) of the system.

  \end{itemize}
To solve the tasks set, an expert ostis system can be
used, which is capable of detecting abuses and anomalies
in the behavior of all participants in the OSTIS Ecosystem
based on continuous monitoring and the introduction of
protocols for the interactions of participants.
\par
The creation and application of expert systems is one
of the important stages in the development of information
technology and information security [12]. Accordingly,
the solution to the problems of ensuring information
security can be obtained based on the use of expert
systems:
\begin{itemize}
    \item it becomes possible to solve complex problems with
the involvement of a new mathematical apparatus
specially developed for these purposes (semantic
networks, frames, fuzzy logic);



    \item the use of expert systems can significantly improve
the efficiency, quality and efficiency of decisions
through the accumulation of knowledge.


  \end{itemize}
\begin{center}
IV. CONCLUSION
\end{center}
For effective information protection of the system at
the present stage, a symbiosis of traditional technologies
\end{multicols}
\newpage
\begin{multicols}{2}
and technologies implemented within the framework of
OSTIS is required. It should also be noted that ensuring
information security based on OSTIS technology is
much easier, because many aspects have already been
implemented at the design stage of the technology itself.
It is important to note that a new generation intelligent
information system is an independent entity that can
consciously, purposefully and constantly take care of
itself, including its own security.
\begin{center}
 REFERENCES
 \end{center}
\begin{itemize}
 \footnotesize{
  \renewcommand{\labelitemi}{[1]}
    \item S. Isoboev, D. Vezarko, and A. Chechel’nitskii, “Intellektual’naya
sistema monitoringa bezopasnosti seti besprovodnoi svyazi na
osnove mashinnogo obucheniya [intelligent system for monitoring
the security of a wireless communication network based on
machine learning],” 
 \textit{Ekonomika i kachestvo sistem svyazi, no.
1(23}. pp. 44–48, 2022.
\renewcommand{\labelitemi}{[2]}
    \item  A. Skrypnikov, V. Denisenko, E. Khitrov, I. Savchenko, and
K. Evteeva, “Reshenie zadach informatsionnoi bezopasnosti s
ispol’zovaniem iskusstvennogo intellekta [solving information
security problems using artificial intelligence],”  \textit{Sovremennye
naukoemkie tekhnologii}. no. 6, pp. 277–281, 7 2021.
\renewcommand{\labelitemi}{[3]}  
\item  V. Chastikova and A. Mityugov,  p. 277–281, 7 2021.
[3] V. Chastikova and A. Mityugov, “Metodika postroeniya sistemy
analiza intsidentov informatsionnoi bezopasnosti na osnove
neiroimmunnogo podkhoda [methodology for building an
information security incident analysis system based on the
neuroimmune approach],” Elektronnyi Setevoi Politematicheskii
Zhurnal «Nauchnye Trudy Kubgtu», no. 1, pp. 98–105, 2022.
\renewcommand{\labelitemi}{[4]}
\item D. D. Abdurakhman, “Iskusstvennyi intellekt i mashinnoe
obuchenie v kiberbezopasnosti [artificial intelligence and machine
learning in cybersecurity],”  \textit{Sovremennye problemy lingvistiki i
metodiki prepodavaniya russkogo yazyka v vuze i shkole,}. Taganrog : no. 34,
pp. 916–921, 2022.;
\renewcommand{\labelitemi}{[5]}
\item ] A. Ostroukh, Intellektual’nye sistemy: monografiya. Krasnoyarsk:
Nauchno-innovatsionnyi tsentr, 2020.
pp. 245–250, 2017, (In Russ.).
\renewcommand{\labelitemi}{[6]}
\item ] A. Baranovich, “Semanticheskie aspekty informatsionnoi
bezopasnosti: kontsentratsiya znanii,” \textit{Istoriya i arkhivy,}, , no.
13(75), pp. 38–58, 2011.
\renewcommand{\labelitemi}{[7]}
\item ] V. Khoang and A. Tuzovskii, “Resheniya osnovnykh zadach
v razrabotke programmy podderzhki bezopasnosti raboty s
semanticheskimi bazami dannykh [solving the main tasks in the
development of a security support program for working with
semantic databases],” \textit{Doklady TUSURa,}. 
 no. 2(28), pp. 121–125,
2013.
\renewcommand{\labelitemi}{[8]}
\item  V. Glenkov, N. Guliakina, I. Davydenko, and D. Shunkevich,
“Semanticheskaya model’ predstavleniya i obrabotki baz znanii
[semantic model for representation and processing of knowledge
bases],” L. Kalinichenko, Y. Manolopulos, N. Skvortsova, and
V. Sukhomlina, Eds. FIC IU RAN, 10 2017, pp. 412–419.
\renewcommand{\labelitemi}{[9]}
\item  A. Dement’ev, “Metriki semanticheskikh dannykh [semantic data
metrics],”  \textit{” Molodoi uchenyi, },no. 24(419), pp. 48–51, 6 2022.
\renewcommand{\labelitemi}{[10]}
\item V. Golenkov, N. Guliakina, I. Davydenko, and A. Eremeev,
“Methods and tools for ensuring compatibility of computer systems,”
in \textit{Open semantic technologies for intelligent systems, }, ser. 4,
V. Golenkov, Ed. BSUIR, Minsk, 2019, pp. 25–52
\renewcommand{\labelitemi}{[11]}
\item V. Druzhinin and D. Ushakov, \textit{Kognitivnaya psikhologiya. Uchebnik dlya vuzov [Cognitive psychology. Textbook for universities].}M.: PER SE, 2002.
\renewcommand{\labelitemi}{[12]}
\item E. Sozinova, “Primenenie ekspertnykh sistem dlya analiza i
otsenki informatsionnoi bezopasnosti [the use of expert systems
for the analysis and evaluation of information security],”   \textit{” Molodoi
uchenyi} : no. 10(33), pp. 64–66, 10 2011. [Online]. Available:
https://moluch.ru/archive/33/3766/
\end{itemize}\\
\begin{center}
 \textbf{\Large{Обеспечение информационной
безопасности Экосистемы OSTIS
 }}
 \end{center}
 \begin{center}
\Large{Чертков В. М., Захаров В. В.}\\

 \end{center}
Большое разнообразие моделей обеспечения информационной безопасности, всё возрастающий объем
данных, которые необходимо анализировать для обнаружения атак на информационные системы, изменчивость
методов атак и динамическое изменение защищаемых
информационных систем, необходимость оперативного
реагирования на атаки, нечеткость критериев обнаружения атак и выбора методов и средств реагирования
на них, нехватка высококвалифицированных специалистов по защите влечет за собой потребность в
использовании методов искусственного интеллекта для
решения задач безопасности.
\par
В статье рассмотрены подходы к использованию
искусственного интеллекта для обеспечения безопасности традиционных информационных систем, особенности обеспечения информационной безопасности
интеллектуальных систем нового поколения и основные
угрозы и принципы, лежащие в основе обеспечения
информационной безопасности ostis-систем.


\ \ \ \ \ \ \ \ \ \ \ \ \ \ \ \ \ \ \ \ \ \ \ \ \ \ \ \ \ \ \ \ \ \ \ \ \ \ \ \ \ \ \ \ \ \ Received 13.03.2023 \ \\ \ \\ \ \\ \ \\ \ \\ \ \\ \ \\ \ \\ \ \\ \ \\ \ \\ 
\end{multicols}
\newpage
\ \\
\begin{center}
\LARGE\textbf{Semantic Approach to Designing Applications
with Passwordless Authentication According to
the FIDO2 Specification}
\large
\par
Anton Zhidovich and Alexei Lubenko and Iosif Vojteshenko and Alexey Andrushevich
\par
Belarusian State University
\par
Minsk, Belarus
\par
{anton.zhidovich, alexeilubenko02}@gmail.com, {voit, andrushevich}@bsu.by
\end{center}\ \\  
\begin{multicols}{2}
\textbf{\textit{Abstract—In} this paper, a semantic approach to designing applications with the FIDO2 specification-based
passwordless authentication using OSTIS technology is
proposed. Obtained results will improve the efficiency of
the component approach to the development of applications
with passwordless authentication, as well as provide the
ability to automatically synchronize different versions of
components, increasing their compatibility and consistency.}
\par
\textbf{\textit{Keywords}—FIDO2 technology, passwordless authentication, OSTIS technology, biometrics}
\begin{center}
I. INTRODUCTION
\end{center}
\par
When building an intelligent system, it’s necessary
to accord special priority to the issue of access to
system resources and the differentiation of user rights.
The key concept here is authentication – a procedure
of identity verification to ensure that the user is the
subject whose identifier he uses. The issue becomes
more complicated when designing different semantically
compatible intelligent systems [1], requiring a unified
authentication apparatus: easy to use and integrate, as
well as the most secure.
\par
The authentication system is only a component, and
therefore the development of a unified approach to
its design is required. There are various authentication
standards, many of which may not provide a high level
of security and, moreover, may be compatible only with
a certain software class or be proprietary.
\begin{center}
II. ANALYSIS OF AUTHENTICATION METHODS


\end{center}
\par
Let’s take a look at the most common authentication
methods. These methods can be encountered both in
everyday life, with the use of messengers, online banking
or other online services, and within corporate systems
where data is accessed by company employees and
delineated according to their position.
\par 
A. \textit{Password-based authentication}
\par
Being the most common because of its ease of
implementation, password-based authentication method is
vulnerable to the most types of attacks: brute-force, range
attacks, dictionary attacks, key-logging, social engineering
such as phishing, man-in-the-middle and replay attacks.
\par
B. \textit{Trusted third-party authentication}
\par
The method is based on the fact that the service
(provider) that owns the user’s data, with his permission,
provides third-party applications with secure access to
this data. The provider is usually a service such as
Google, GitHub, Facebook or Twitter. The most common
implementation is the OAuth 2.0 protocol.
\par
The OAuth 2.0 specification defines a protocol for
delegating user authentication to the service that hosts a
user account and authorising third-party applications to
access that user account [2].
\par
In [2] the main participants of OAuth 2.0 authentication
and their interaction are described. Although this method
is one of the most user-friendly and, moreover, implemented in most online resources, it is still vulnerable
to a man-in-the-middle attack, which is a common and
effective way to gain unauthorised access to a system.
\par
C. \textit{One-time password (OTP)}
\par
OTP, which is used in many systems as a second
or first authentication factor, can be a number or some
string that is generated for a single login process. When
authenticating, the OTP can be sent to the user via SMSmessage, push notification or in a special application.
The most secure tool for generating one-time passwords
is a token (software, such as Google Authenticator, or
hardware).
\par
OTP quickly becomes invalid, which provides resistance to replay attacks. However, most attacks on
authentication systems with OTP target the way the user
receives it. For example, OTP transmitted via SMS can be
intercepted by software such as FlexiSPY or Reptilicus.
\par
One-time passwords are protected against phishing in
the classic sense: users cannot reveal long-term credentials.
However, the man-in-the-middle attack can be used to
retrieve a currently valid one-time password.
\par
D. \textit{Passwordless authentication methods}
\par
Passwordless authentication allows a user to access
an information system without entering a password or
answering security questions. Instead, the user provides
\end{multicols}
\end{document}
