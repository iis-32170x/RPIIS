\documentclass{article}
\usepackage[utf8]{inputenc}
\usepackage[left=17mm, top=17mm, right=17mm, bottom=25mm, nohead]{geometry}
\usepackage{enumitem}
\usepackage{multicol}
\usepackage{ulem}
\usepackage{mathtools}
\setlist[itemize]{itemsep=0pt, parsep=0pt, partopsep=0pt, topsep=1.4pt}
\setcounter{page}{307}
\begin{document}
\begin{multicols}{2}

    \begin{description}[leftmargin=!, labelwidth=0.7cm, itemsep=-1.5mm]
leakage; for this, duplication of means and security
measures is allowed;
    \end{description}
\par
\vspace{-0.4cm} \begin{itemize}
    \item the principle of universality - security measures
should block the path of threats, regardless of the
place of their possible impact;
    \item the principle of planning – planning should be carried
out by developing detailed action plans to ensure
the information security of all components of the
system for the provision of public services;
    \item the principle of centralized management – within
a certain structure, the organized and functional
independence of the process of ensuring security in
the provision of public services should be ensured;
    \item the principle of purposefulness – it is necessary to
protect what must be protected in the interests of a
specific goal;
    \item the principle of activity - protective measures to
ensure the safety of the service delivery process
must be implemented with a sufficient degree of
persistence;
    \item the principle of service personnel qualification –
maintenance of equipment should be carried out
by employees who are trained not only in the
operation of equipment, but also in technical issues
of information security;
    \item the principle of responsibility - the responsibility
for ensuring information security must be clearly
established, transferred to the appropriate personnel
and approved by all participants as part of the
information security process.
\end{itemize}
    \begin{center} III.  THE PRINCIPLES UNDERLYING THE INFORMATION
SECURITY OF OSTIS SYSTEMS
    \end{center}
The OSTIS ecosystem is a collective of interacting:

\begin{itemize}
    \item ostis-systems;
    \item  users of ostis systems (end users and developers);
    \item other computer systems that are not ostis-systems,
but are additional information resources or services
for them.
\end{itemize}

\par 
The core of OSTIS technology includes the following
components:

\begin{itemize}
    \item semantic knowledge base OSTIS, which can describe
any kind of knowledge, while it can be easily
supplemented with new types of knowledge;
    \item OSTIS problem solver based on multi-agent approach. This approach makes it easy to integrate
and combine any problem solving models;
    \item ostis-system interface, which is a subsystem with its
own knowledge base and problem solver.
\end{itemize}
\par
The presented architecture of the OSTIS Ecosystem
implements:
    \begin{itemize}
        \item  all knowledge bases are united into the Global
Knowledge Base, the quality of which (logicality,
correctness, integrity) is constantly checked by many
agents. All problems are described in a single knowledge base, and specialists are involved to
eliminate them, if necessary;
    \item each application associated with the OSTIS Ecosystem has access to the latest version of all major
OSTIS components, components are updated automatically;
    \item each owner of the OSTIS Ecosystem application can
share a part of their knowledge for a fee or for free.
\end{itemize}
    \par
It is important to note that information security is
closely related to the architecture of the built system: a
well-designed and well-managed system is more difficult
to hack. Therefore, it is very important to develop an
information security system at the stage of designing
the architecture and structure of a future next-generation
intelligent system.
    \par 
The OSTIS Ecosystem is a community where ostis
systems and users interact, where rules must be established and controlled. Illegal and destabilizing actions by
all members of the community should not be allowed.
The user cannot directly interact with other ostis systems,
but only through a personal agent. This agent stores all
personal data of the user and access to them should be
limited.
    \par 
In the OSTIS Ecosystem, all agents must be identified.
It should be noted that the personal user agent in the
Ecosystem solves the problem of identifying the user
himself.
    \par
    In the considered OSTIS Ecosystem, it is required to
organize information security at each of the levels of
interaction: data exchange, data access rights, authentication of Ecosystem clients, data encryption, obtaining
data from open sources, ensuring the reliability and
integrity of stored and transmitted data, monitoring the
violation of communications in knowledge base, tracking
vulnerabilities in the system.
\vspace{0.7cm}
\par

\noindent \textit{\textbf{threat in ostis-system}}
 \begin{description}[leftmargin=!, labelwidth=1cm, itemsep=-1.5mm]
   \item[$\supset$] \textit {threat. breach of confidentiality of information}
   \hspace{0.5cm} $\Rightarrow$  \hspace{0.5cm} \textit{explanation}:* 
   \par \vspace{-0.1cm} \hspace{1cm} \textbf{[}unauthorized access to read information\textbf{]}
    \item[$\supset$] \textit {threat. violation of the integrity of information} \\
   \vspace{0.1cm} \hspace{-0.23cm}  $\Rightarrow$  \hspace{0.5cm} \textit{explanation}:* 
     \par
      \begin{description}[leftmargin=!, labelwidth=1cm, itemsep=-1.5mm]
     \item[ ] \vspace{-0.4cm} \textbf{[}unauthorized or erroneous change, distortion or destruction of information, as well
as unauthorized impact on technical and
software information processing tools\textbf{]}
\end{description}
 \item[$\supset$] \vspace{-0.2cm} \textit {threat. accessibility violation} \\
   \vspace{0.1cm} \hspace{-0.23cm}  $\Rightarrow$  \hspace{0.5cm} \textit{explanation}:* 
   \par 
   \begin{description}[leftmargin=!, labelwidth=1cm, itemsep=-1.5mm]
   \item \vspace{-0.5cm} \textbf{[}blocking access to the system, its individual components, functions or information,
as well as the impossibility of obtaining
information in a timely manner (unacceptable delays in obtaining information)\textbf{]}
\end{description}
 \item[$\supset$] \vspace{-0.2cm} \textit {threat. violation of semantic compatibility} \\
   \vspace{0.1cm} \hspace{-0.23cm}  $\Rightarrow$  \hspace{0.5cm} \textit{explanation}:* 
\end{description}
\end{multicols}
\end{document}
