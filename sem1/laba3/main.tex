\documentclass{article}
\usepackage[utf8]{inputenc}
\usepackage[main=russian,english]{babel}
\usepackage{graphicx}
\usepackage{multicol}
\usepackage{titlesec}
\usepackage{enumitem}
\usepackage{capt-of}
\usepackage[left=15mm, top=15mm, right=15mm, bottom=15mm, nohead, nofoot]{geometry}
\usepackage{titling}

\setlist[itemize]{noitemsep}
\title{\LARGE\textbf{An Automated Approach to Checking User\\ Knowledge Levels in Intelligent Tutoring\\ Systems}}
\date{} 
\setlength{\droptitle}{-1cm}
\titleformat{\section}[block]{\normalfont\bfseries\filcenter}{}{0em}{}
\titlespacing*{\section}{0pt}{\baselineskip}{\baselineskip}

\begin{document}
\begin{multicols}{2}
\setcounter{page}{267} 
\begin{enumerate}[noitemsep]
    \item Open-source and free: Protégé is a free and opensource ontology editor, which makes it accessible to anyone who wants to use it.
    \item User-friendly interface: Protégé has a user-friendly interface that allows users to easily create, edit, and manipulate ontologies.
    \item Strong community support: Protégé has a strong community of academic government, and corporate users who use it to build knowledge-based
solutions in various areas such as bio-medicine, e-commerce, and organizational modeling. This support ensures that Protégé is continually updated and maintained.
    \item  Customizable: Protégé allows users to customize
their ontologies by adding new classes and properties and modifying existing ones.
    \item  Integration with other tools: Protégé can be integrated with other tools such as reasoners and
visualization plugins, which makes it a powerful
tool for ontology development.
    \item Reasoning capabilities: Protégé has built-in reasoning capabilities that allow users to check the
consistency of their ontologies and detect errors.
    \item Overall, Protégé is a powerful and flexible tool
for creating and managing ontologies, with a userfriendly interface, strong community support, and
a range of customizable features.
\end{enumerate}
\vspace{7pt}
\begin{center}
 \chapter{IV. ONTOLOGICAL MODEL} 
\end{center}
\vspace{7pt}
\par The domain ontological model was built using
the Protégé editor. We have identified the following three classes: "Documents", "Events", "People".
The "TextDocuments", "Pictures", "Video" and "Audio".
Class "People" — subclasses "Graduates", "Directors",
"Mentors". The "Events" class is subclasses corresponding to the reign of one or another director of the corps.
\par The structure of the upper level of the ontology "History of the Polotsk Cadet Corps" is shown in Figure 1.
\vspace{7pt}
\begin{center}
\chapter{V. CREATION OF THE INTELLIGENT INFORMATION
AND REFERENCE SYSTEM}
\end{center}
\vspace{7pt}
\par We plan to complete the creation of the Intellectual Information and Reference System "History of the Polotsk
Cadet Corps" using the OSTIS technology stack. We list
the advantages of this approach, which guarantee the
success of the completion of the project as a whole.
\begin{enumerate}[noitemsep]
    \item  Any OSTIS — system can be easily supplemented
with new knowledge or new methods for solving
problems. This allows you to quickly and easily
repurpose the developed intelligent system, reorienting it to a new range of tasks to be solved.
    \item OSTIS — the system is focused on the reuse of
the developed components. Thanks to a single and
universal SC code, a library of typical components
can be created, the use of which in the design
\columnbreak
\setcounter{figure}{1}
\afterpage{\clearpage}
\begin{center}
  \includegraphics[width=\linewidth]{схема.jpeg}
  \captionof{figure}{The structure of the upper level of the ontology "History of the Polotsk Cadet Corps"}
  \label{fig:структура}
\end{center}
process can reduce development time by 40-60 percent.
\item OSTIS — the system is reflexive, i.e. can analyze
itself, due to the fact that it is fully described
using the SC code. Reflexivity is one of the most
important qualities of intelligent systems.
    \item Due to the fact that the design of OSTIS —
the system is reduced to the construction of its
SC-model, it is completely platform-independent,
and can be implemented both in software and in
hardware.
    \item Hardware implementation involves the creation of
a new generation of computing devices — semantic
computers.
\end{enumerate}
\begin{center}
\chapter{VI. CONCLUSION}
\end{center}
\par In conclusion, we note the following. Ontological
modeling is an important method for developing intelligent information and reference systems. By focusing on relationships between concepts and entities, this
approach ensures that knowledge is represented in an
intuitive and machine-readable way.
\par With the help of ontological modeling, one can define
a subject area, develop a conceptual model, create and
populate an ontology with data, and finally put the
ontology into action. Although it can be a complex
process, the benefits of ontology modeling are clear: it
provides a foundation for building powerful applications
that address a variety of application problems.
\par An effective tool for building ontologies is the Protégé
editor. It allows you to create classes, slots and instances,
and also provides an easy way to modify ontologies without creating inconsistent data and knowledge. Protégé
can be used for practical applications, such as creating
ontologies for intelligent information retrieval systems,
\end{multicols}
\clearpage
\begin{multicols}{2}
as well as for educational purposes, such as creating
ontology models for e-learning.
\par Note also that Protégé can be easily integrated with
other software used to work with ontologies. Based on
the ontology created by Protégé tools, using the OSTIS
technology stack, you can quickly create an Intelligent
Information Retrieval System or a digital archive of an
organization.
\par Returning to the IIRS "History of the Polotsk Cadet
Corps", we note that the electronic resource created as
a result of the above steps can be used in at least
three organizations. In the Euphrosyne Polotskaya State
University of Polotsk — as an exhibit of the Museum
of the History of Science and Education of Polotsk, in
the Polotsk Cadet Corps — as an exhibit of the corps
museum being created, as well as in the exposition of
the Polotsk National Historical and Cultural MuseumReserve . In addition, the resource will certainly find
application in the educational process of Polotsk State
University.
\begin{center}
\chapter{ACKNOWLEDGMENT}
\end{center}
\par The author expresses his deep gratitude to Professor
V. V. Golenkov, thanks to whose advice and support I
decided to write this article.
\begin{center}
\chapter{REFERENCES}
\end{center}
{\footnotesize
\begin{enumerate}[label={[}\footnotesize\arabic*{]}, noitemsep]
\item V. V. Golenkov and N. A. Guljakina Cemanticheskaja tehnologija
komponentnogo proektirovanija sistem, upravljaemyh znanijami.
\textit{Otkrytye semanticheskie tekhnologii proektirovaniya intellektual’nykh system [Open semantic technologies for intelligent systems]}, Minsk, BGUIR, 2015, pp. 57–78 (In Russ.)
\item A. Maedche, B. Motik, L. Stojanovic, R. Studer, R. Volz Ontologies for enterprise knowledge management,\textit{ IEEE Intelligent
Systems,} vol. 18, No. 2, IEEE, 2003, pp.26–33.
\item R. S. Oktari, K. Munadi, R. Idroes, H. Sofyan Knowledge management practices in disaster management: Systematic review.
\textit{International Journal of Disaster Risk Reduction}, vol. 51, Elsevier,
2020, pp. 10–18.
\item V. A. Carriero, A. Gangemi, M. Aldo, M. L. Mancinelli, L.
Marinucci, A. G. Nuzzolese, V. Presutti, C. Veninata. ArCo: The
Italian cultural heritage knowledge graph. The Semantic Web–
ISWC 2019: 18th International Semantic Web Conference, Auckland, New Zealand, October 26–30, 2019, Proceedings, Part II 18,
Springer, 2019, pp. 36–52.
\item M. L. Turco, M. Calvano, E. Giovannini Data modeling for museum collections. \textit{International Archives of the Photogrammetry,
Remote Sensing and Spatial Information Sciences}, vol. 42, No. 2,
2019
\item A. Vlachos, M. Perifanou, A. Economides A review of ontologies
for augmented reality cultural heritage applications.\textit{ Journal of
Cultural Heritage Management and Sustainable Development,}
Emerald Publishing Limited, 2019.
\item M. N. Buharov Informacionnaja sistema dlja ontologicheskogo
modelirovanija predmetnyh oblastej [Information system for ontological modeling of subject areas]. \textit{Journal of Cultural Heritage
Management and Sustainable Development,} Informatika, vol. 19,
No. 2, 2022 (In Russ.)
\item S. P. Vorona, E. O. Savkova Ispolzovanie ontologicheskogo
modelirovanija pri razrabotke intellektual’noj sistemy dostupa k
uchebno-metodicheskoj informacii [The use of ontological modeling in the development of an intelligent system for access to
educational and methodological information]. Informatika, upravljajushhie sistemy, matematicheskoe i komp’juternoe modelirovanie
(IUSMKM-2020), 2020, pp. 208–213(In Russ.)
\end{enumerate}
\columnbreak
\begin{enumerate}[label={[}\footnotesize\arabic*{]}, noitemsep, resume]
\item E. Kalashnikov, V. Pavlenko Primenenie ontologicheskogo modelirovanija v raznyh otrasljah [Application of ontological modeling
in different industries], XXV Nauchno-tehnicheska konferencija s
mezhdunarodnym uchastiem sbornik dokladov, P. 97 (In Russ.)
\item J. A. Balyberdin Ontologicheskoe modelirovanie predmetnyh
znanij v vuzah, konstruktorskih bjuro i nauchno-issledovatel’skih
institutah [Ontological modeling of subject knowledge in universities, design bureaus and research institutes], 2022 (In Russ.)
\item B. Reitemeyer, H.-G. Fill Enterprise, Business-Process and Information Systems Modeling: 20th International Conference, BPMDS 2019, 24th International Conference, EMMSAD 2019, Held
at CAiSE 2019, Rome, Italy, June 3–4, 2019, Proceedings 20,
Springer, 2019, pp. 212–226
\item ] Z. A. A. Salam, R. A. Kadir, A. Azman Ontology Merging
Using PROTÉGÉ–a Case Study. \textit{Journal of Information System
and Technology Management}, JISTM, 2021
\item  C. Shimizu, K. Hammar, P. Hitzler Modular graphical ontology
engineering evaluated. The Semantic Web: 17th International Conference, ESWC 2020, Heraklion, Crete, Greece, May 31–June 4,
2020, Proceedings 17, Springer, 2019, pp. 20–35
\item E. V. Glazyrin, T. E. Sohor Polockij kadetskij korpus v
otkrytyh pis’mah, vypushhennyh «Obshhestvom vzaimopomoshhi
polochan» v nachale XX veka [Polotsk cadet corps in open letters
issued by the Polokhan Mutual Aid Society at the beginning of
the 20th century], 2019 (In Russ.)
\item  S. G. Ljutko Stanovlenie i razvitie voennogo obrazovanija na
zemljah Belarusi v XVIII–nachale XX veka [Formation and development of military education on the lands of Belarus in the XVIII
— the beginning of the XX century], Minskij gosudarstvennyj
lingvisticheskij universitet, 2021 (In Russ.)
\item  S. Poljakov and others Polockij kadetskij korpus. Istorija v licah
[Polotsk cadet corps. History in faces], Matjeryjaly navukovapraktychnaj kanferjencyi: pa vynikah navukova-dasledchaj raboty
2008 g, NPIKZ, 2009, pp.48–59 (In Russ.)
\item V. P. Vikent’ev Polockij kadetskij korpus. Istoricheskij ocherk 75-
letija ego sushhestvovanija [Polotsk cadet corps. Historical sketch
of the 75th anniversary of its existence], Polock: Tip. XV Kljachko,
1910 (In Russ.)
\item V. V. Golenkov, N. A. Gulyakina, I. T. Davydenko, D. V.
Shunkevich Semantic technologies of intelligent systems design
and semantic associative computers. \textit{Doklady BGUIR}, 2019, No. 3,
pp. 42–50
\item V. V. Golenkov, N. A. Gulyakina, V. A. Golovko, V. V. KrasnoproshinMethodological problems of the current state of works
in the field of Artificial intelligence.\textit{ Otkrytye semanticheskie
tekhnologii proektirovaniya intellektual’nykh system [Open semantic technologies for intelligent systems]}, 2021, vol. 5, pp. 17–32
\end{enumerate}
}
\section{Создание комплексного цифрового\\
архива полоцкого кадетского корпуса:\\
пример проектирования\\
интеллектуальной
информационно-справочной системы}
\vspace{-0.5em}
\begin{center}
Оськин А. Ф.
\end{center}

{\small
\par На примере проектирования и построения Интеллектуальной информационно-поисковой системы (ИИПС) «Полоцкий кадетский корпус» описаны основные этапы создания ИИПС. Кратко рассмотрены принципы онтологического моделирования и инструменты, используемые для этих
целей. На основе анализа предметной области строится
онтологическая модель, которая может быть преобразована
в работающую ИИПС с использованием стека технологий
OSTIS. Описаны преимущества технологий OSTIS для решения задач подобного рода.
}
\begin{flushright}
Received 27.03.2023
\end{flushright}
\end{multicols}
\clearpage
\maketitle
\begin{center}
\vspace{-2,5cm}
\large Wenzu Li\\ \textit{Belarussian State University Informatics and Radioelectronics, Минск, Беларусь} \\
Минск, Беларусь\\
lwzzggml@gmail.com
\end{center}
\begin{multicols}{2}
\textbf{\textit{Abstract}}---\textbf{This article is dedicated to the issue of automating the implementation of rapid testing of user knowledge in new generation intelligent tutoring systems. A semantic-based approach to automating the entire process from test question generation and test paper generation to the automatic verification of user answers and the automatic scoring of test papers is described in detail in this article.}

\par \textbf{\textit{Keywords}}---\textbf{testing user knowledge level, test question generation, user answer verification, intelligent tutoring systems, test paper generation, automatic scoring of test papers}
\vspace{10pt}
\begin{center}
\chapter{I. INTRODUCTION}
\end{center}
\par Educators have long shared a common desire to use
computers to automate teaching and learning services. In
recent years, with the development of artificial intelligence
technology, this wish is likely to become a reality. The
most representative product combining artificial intelligence and education is the intelligent tutoring system
(ITS), which can not only improve the learning efficiency
of users, but also ensure the fairness and impartiality of
the education process [9].
\par Automatic generation of test questions and automatic
verification of user answers are the most basic and
important functions of ITS. Using these two functions
in combination will enable the entire process from the
automatic generation of test questions to the automatic
scoring of the user test papers. This will not only greatly
reduce the repetitive work of educators, but will also
reduce the cost of user learning, thus providing more
users with the opportunity to learn various knowledge
[1], [2], [7].
\par Although in recent years, with the development of
technologies such as the semantic web, deep learning and
natural language processing (NLP), several approaches
have been proposed for the automatic generation of test
questions and the automatic verification of user answers,
these approaches have the following main disadvantages:
\begin{itemize}
    \item existing approaches to generating test questions allow only the simplest objective questions to be generated;
    \item some existing approaches (for example, keyword matching and probability statistics) to verifying user
    \columnbreak
    \item[] answers to subjective questions do not consider the semantic similarity between answers;
    \item methods that use semantic to verify user answers to subjective questions can only calculate similarity between answers with simple semantic structures [7], [8], [10].
\end{itemize}
\par Objective questions usually have a unique standard answer. In this article, objective questions include: multiplechoice questions, fill in the blank questions and judgment
questions. Objective questions differ from subjective
questions, which have more than one potential correct
answer. Subjective questions in this article include: definition explanation questions, proof questions and problemsolving task.
\par Therefore, based on existing methods and OSTIS Technology, an approach to developing a universal subsystem
for automatic generation of test questions and automatic
verification of user answers in tutoring systems developed
using OSTIS Technology (Open Semantic Technology
for Intelligent Systems) is proposed in this article [1],
[2], [5]. The universality of the subsystem means that the
developed subsystem can be easily transplanted to other
ostis-systems (system built using OSTIS Technology). The
developed subsystem allows the use of the knowledge
bases of the ostis-systems to automatically generate
various types of test questions and automatically verify
the completeness and correctness of user answers based
on the semantic description structures of the knowledge.
The discrete mathematics ostis-system will be used as
demonstration systems for the developed subsystem.

\begin{center}
\chapter{II. EXISTING APPROACHES AND PROBLEMS}
\end{center}
\vspace{-4pt}
\textit{A. Automatic generation of test questions}
\vspace{5pt}
\par Approach to automatic generation of test questions
mainly studies how to use electronic documents, text
corpus and knowledge bases to automatically generate
test questions. Among them, the knowledge base stores
highly structured knowledge that has been filtered, and
with the development of semantic networks, using the
knowledge base to automatically generate test questions
has become the most important research direction in
\end{multicols}
\end{document}
