\documentclass{article}
\usepackage[T2A]{fontenc}
\usepackage[utf8]{inputenc}
\usepackage[english,russian]{babel}
\usepackage{enumitem}
\usepackage{multicol}
\usepackage{ulem}

\usepackage{geometry}
\geometry{left=17mm, right=17mm, top=17mm, bottom=9mm, nohead, nofoot}
\setcounter{page}{231}
\begin{document}
\begin{multicols}{2}
\noindent into the kernel and placed in packages, exceeds 5500.
The kernel (not in full composition) uses MATLAB
and MathCad (starting from version 14, symbolic kernel
MuPAD is used).
\par
The main stages of development, additions in the
Maple versions can be traced in [20], but it must be
stated that machine learning, artificial intelligence, and
data mining are not yet priorities for the system.
\par
\textbf{Wolfram Mathematica.} Mathematica – is a computer
algebra system developed by Wolfram Research company. It is one of the most powerful and widely used
integrated multimedia-technology software package [6],
[22]–[25]. Mathematica is recognized as a fundamental
advancement in computer-aided design. It is one of
the world’s largest software packages by volume of
modules and contains many new algorithms, as well
as many unique developments. Mathematica lets users
use virtually every analytic and numeric option, and it
supports databases, graphics, and sound. Mathematica
lets you work, analyze, manipulate, and graph almost
any function of pure and applied mathematics. The
system provides calculations with any specified accuracy; construction of two- and three-dimensional graphs,
their animation, drawing geometric figures; importing,
processing, exporting images and sound.
\par
Mathematica has evolved from a program used primarily for mathematical and technical calculations to a
tool widely used in various other areas [22], [23]. It is
recognized among specialists as a development platform
that fully integrates computation into the workflow from
start to end, seamlessly guiding the user from initial ideas
to deployed custom and industrial solutions.
\par
Mathematica has a built-in Wolfram Language, including tools for creating programs and user interfaces,
connecting external dlls, and parallel computing. The
system’s programming language is a typical interpreter;
it’s not designed to create executable files, but it incorporates the best of such programming languages
as BASIC, Fortran, Pascal, and C. The Mathematica
programming language supports all known paradigms:
functional, structural, object-oriented, mathematical, logical, recursive, and more. It also includes visual-oriented
programming tools based on the use of mathematical
symbol templates, such as integral, summation, product,
etc.; this language exceeds the usual general-purpose
programming languages in its ability to perform mathematical and scientific computations.
\par
Like all computer algebra, Mathematica is a type
of software tool designed to manipulate mathematical
formulas. Its main purpose is to automate the often
tedious and in many cases difficult algebraic transformations. User works in the system with notebooks - NB
documents, each document contains at least one section
(cell). An explanation of the preference adopted here is
a comparison with MS Excel, where the term cell is
steadily and universally used. Those who have experience
with Excel and Mathematica understand the difference
and that in MS Excel it is cells, and in Mathematica
notebooks it is more general objects.
\par
NB documents can be opened, viewed, edited, saved,
executed in their entirety, or individual cells. Notebook’s
interface contains many palettes (menus) and graphical
tools for creating, editing, viewing documents, sending
and receiving data to and from the core. Notebook
includes one or several cells that can be grouped together
as needed. Each cell contains at least one line of text or
formulas, a digital audio or video object. Notebooks can
be edited as text in any editor or in the Mathematica
interface. The kernel performs the computations and
can be run on the same computer as the interface,
or on another computer connected through a network.
Typically, the kernel is started when the computation
begins. Cellss in Mathematica can be roughly divided
into input and result (output) cells. In the input cells, the
user enters or places commands, comments, multimedia
objects, and they can be executable or otherwise; the
executable cells are processed – the system returns results
and displays them in output cells.
\par
All versions of Mathematica include a powerful reference database, and the built-in Help, Documentation
Center is an example of an NB document in itself.
Without interrupting work on modules, you can clarify
any function, option, directive, or service word; explore
the capabilities of “live” examples to get and document
results; and embed examples or code snippets from
examples in your own code.
\par
\underline{From the chronology of Mathematica versions.} The
first release of Mathematica was in June 1988, the
basic concept being to create once and for all one
system for different computations in a consistent and
unified way. The basis for this was the creation of a
new symbolic computer language for controlling, with
a minimum number of inputs, the large number of
objects involved in technical computation. Since its
inception, all Wolfram Research Inc. developments have
been regularly ranked first among IT achievements,
highlighted by the media.
\par
Release dates, additions, and updates to Mathematica
are fully reflected in a number of publications and
websites, e.g. [22], [23]. Experts note that the list of updates to Wolfram Mathematica reflects many completely
new advances that have found application, development
in other systems, and information technology. Wolfram
Research Inc. developments are mostly characterized by
interface continuity and the ability to use source code
from previous versions.
\par
\underline{About Mathematica features.} A complete list of capabilities would require several times as much space as
this presentation allows. For example, manual [24] has
over 600 pages of content, but in fact it only outlines
\end{multicols}
\end{document}
